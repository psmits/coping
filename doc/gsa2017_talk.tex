\documentclass[aspectratio=169]{beamer} 
\usepackage{amsmath,amsthm}
\usepackage{graphicx,microtype,parskip}
\usepackage{caption,subcaption,multirow}
\usepackage{attrib}

\frenchspacing

\usetheme{default}
\usecolortheme{whale}

\setbeamertemplate{navigation symbols}{}

\setbeamercolor{title}{fg=blue,bg=white}

\setbeamercolor{block title}{fg=white,bg=gray}
\setbeamercolor{block body}{fg=black,bg=lightgray}

\setbeamercolor{block title alerted}{fg=white,bg=darkgray}
\setbeamercolor{block body alerted}{fg=black,bg=lightgray}

\usepackage{etoolbox}
\newcommand{\zerodisplayskips}{%
  \setlength{\abovedisplayskip}{0pt}%
  \setlength{\belowdisplayskip}{0pt}%
  \setlength{\abovedisplayshortskip}{0pt}%
  \setlength{\belowdisplayshortskip}{0pt}}
  %\appto{\normalsize}{\zerodisplayskips}
  %\appto{\small}{\zerodisplayskips}
  %\appto{\footnotesize}{\zerodisplayskips}


\title{Modeling changes to the functional composition of North American mammal diveristy}
\subtitle{multi-level dynamics of a regional species pool}
\author{Peter D Smits}
\institute{Department of Integrative Biology, University of California -- Berkeley}
%\titlegraphic{
%  \includegraphics[width=3cm,height=3cm,keepaspectratio=true]{figure/paleodb}
%  \hspace*{0.3\paperwidth}
%  \includegraphics[width=4cm,height=4cm,keepaspectratio=true]{figure/iblogo3}
%}
\date{}

\begin{document}

\begin{frame}
  \maketitle
\end{frame}


\begin{frame}
  \begin{alertblock}{Question}
    \alert{Why} do the relative diversities of functional groups change within a species pool?
    \begin{itemize}
      \item function of \alert{species traits} and \alert{environmental context}
    \end{itemize}
  \end{alertblock}
\end{frame}


\begin{frame}
  \frametitle{Eco-cube and functional groups}

  \begin{center}
    \includegraphics[height=0.775\textheight,width=\textwidth,keepaspectratio=true]{figure/ecocube}
  \end{center}

  \tiny{\attrib{Bambach \em{et al.}, 2007, \em{Palaeontology}}}
\end{frame}


\begin{frame}
  \frametitle{Species pool concept}

  \begin{center}
    \includegraphics[height=0.775\textheight,width=\textwidth,keepaspectratio=true]{figure/schemske_pool}
  \end{center}

  \tiny{\attrib{Mittelbach and Schemske, 2015, \em{TREE}}}
\end{frame}

\begin{frame}
  \frametitle{Structured, multi-level data in biology}
  \begin{center}
    \includegraphics[width = 0.45\textwidth,height = 0.775\textheight,keepaspectratio = true]{figure/ovaskainen_data_sm}
    \hspace*{0.1\textwidth}
    \includegraphics[width = 0.45\textwidth,height = 0.775\textheight,keepaspectratio = true]{figure/ovaskainen_dag_sm}
  \end{center}

  \tiny{\attrib{Ovaskainen \textit{et al.} 2017 \em{Ecology Letters}}}
\end{frame}

\begin{frame}
  \frametitle{Cenozoic mammals of North America}
  \begin{center}
    \includegraphics[height=0.775\textheight,width=\textwidth,keepaspectratio=true]{figure/aom}
  \end{center}

  \tiny{\attrib{Rudolph Zallinger}}
\end{frame}


\begin{frame}
  \frametitle{Conceptualizing the knowns and unknows}
  \begin{center}
    \includegraphics[width=0.8\textwidth,height=\textheight,keepaspectratio=true]{figure/problem_concept}
  \end{center}
\end{frame}

\begin{frame}
  \frametitle{Covariates of interest, temporal structure}

  \begin{center}
    \large{species occurrence (\(\sim\)1400 species) per NALMA}
  \end{center}

  \vspace*{0.05\textheight}

  \begin{columns}
    \begin{column}[T]{0.3\textwidth}
      \textbf{functional group}
      \begin{itemize}
        \small{
        \item dietary category: \\\begin{tiny}carnivore, herbivore, insectivore, omnivore\end{tiny}
        \item locomotor category: \\\begin{tiny}arboreal, digitigrade, fossorial, plantigrade, scansorial, unguligrade\end{tiny}
        }
      \end{itemize}
  
      \vspace*{0.01\textheight}

      \textbf{observation}
      \begin{itemize}
        \item individual-level: species
          \begin{itemize}
            \small{
            \item functional group
            \item mean mass
            }
          \end{itemize}
        \item time of observation
      \end{itemize}
    \end{column}
    \begin{column}[T]{0.3\textwidth}
      \textbf{origination}
      \begin{itemize}
        \item individual-level: species
          \begin{itemize}
            \small{
            \item functional group
            \item taxon order
            \item mean mass
            }
          \end{itemize}
        \item group-level: time
          \begin{itemize}
            \small{
            \item temperature est Mg/Ca
            \item plant phase \\\tiny{(Pa-Eo, Eo-Mi, Mi-Pl)}
            }
          \end{itemize}
      \end{itemize}
    \end{column}
    \begin{column}[T]{0.3\textwidth}
      \textbf{survival}
      \begin{itemize}
        \item individual-level: species
          \begin{itemize}
            \small{
            \item functional group
            \item taxon order
            \item mean mass
            }
          \end{itemize}
        \item group-level: time
          \begin{itemize}
            \small{
            \item temperature est Mg/Ca
            \item plant phase \\\tiny{(Pa-Eo, Eo-Mi, Mi-Pl)}
            }
          \end{itemize}
      \end{itemize}
    \end{column}
  \end{columns}

\end{frame}

\begin{frame}
  \frametitle{Conceptualizing the analysis}
  \begin{center}
    \includegraphics[width=0.8\textwidth,height=\textheight,keepaspectratio=true]{figure/paleo_fourth_corner}
  \end{center}
\end{frame}


\begin{frame}
  \frametitle{Hidden Markov Model with absorbing state}
  \begin{block}{Jolly-Seber CMR/Restricted occupancy model}
    \begin{align*}
      y_{i, t} &\sim \text{Bernoulli}(z_{i, t} p_{i, t}) \\
      z_{i, t = 1} &\sim \text{Bernoulli}(\phi_{i, t = 1}) \\
      z_{i, t} &\sim \text{Bernoulli}\left(z_{i, t - 1} \pi_{i,t} + \sum_{x = 1}^{t}(1 - z_{i, x}) \phi_{i, t}\right)
    \end{align*}
    \begin{scriptsize}
      \(y\) observed state; \(z\) estimated state.

      \(p\) observation; \(\phi\) origination; \(\pi\) survival.

      \(i\) in \(N\); \(t\) in \(T\).
    \end{scriptsize}
  \end{block}
\end{frame}

\begin{frame}
  \frametitle{Modeling the probabilities; individual-level}
  \begin{block}{Multi-level logistic regression}
    \begin{align*}
      p_{i, t} &\sim \text{logit}^{-1}(b_{t} + e_{j[i]} + \beta^{p} mass_{i}) \\
      \phi_{i, t} &\sim \text{logit}^{-1}(f^{\phi}_{j[i], t} + o^{\phi}_{k[i]} + \beta^{\phi} mass_{i}) \\
      \pi_{i, t} &\sim \text{logit}^{-1}(f^{\pi}_{j[i], t} + o^{\pi}_{k[i]} + \beta^{\pi} mass_{i})
    \end{align*}
    \begin{scriptsize}
      observation: \(b_{t}\) time-varying intercept; \(e_{j[i]}\) functional group eff; \(\beta^{p}\) mass eff.

      origination: \(f^{\phi}_{j[i], t}\) time/FG-varying intercept; \(o^{\phi}_{j[i]}\) order eff; \(\beta^{\phi}\) mass eff.

      survival: \(f^{\pi}_{j[i], t}\) time/FG-varying intercept; \(o^{\pi}_{j[i]}\) order eff; \(\beta^{\pi}\) mass eff.
    \end{scriptsize}
  \end{block}
\end{frame}

%\gamma^{j = 1}_0 + \gamma^{j = 1}_1 phase_{2} + \gamma^{j = 1}_{2} phase_{3} + \gamma^{j = 1}_{3} temp_{t} \\
\begin{frame}
  \frametitle{Modeling the probabilities; group-level}
  \begin{block}{Multivariate regression of time/FG-varying intercept}
    \begin{align*}
      f^{\phi} &\sim \text{MVN}\left(
      \begin{matrix}
        U \gamma^{\phi}_{j = 1} \\
        %U_{t, \_} \gamma^{\phi}_{j = 2} \\
        \vdots \\
        U \gamma^{\phi}_{j = J}
      \end{matrix}, 
      \text{diag}(\tau_{f^{\phi}}) \Omega_{f^{\phi}} \text{diag}(\tau_{f^{\phi}}) \right) \\
      f^{\pi} &\sim \text{MVN}\left(
      \begin{matrix}
        U \gamma^{\pi}_{j = 1} \\
        %U_{t, \_} \gamma^{\pi}_{j = 2} \\
        \vdots \\
        U \gamma^{\pi}_{j = J}
      \end{matrix}, 
      \text{diag}(\tau_{f^{\pi}}) \Omega_{f^{\pi}} \text{diag}(\tau_{f^{\pi}}) \right)
    \end{align*}
    \begin{scriptsize}
      \(U\) matrix group-level covariates; \(\gamma^{\phi}\), \(\gamma^{\pi}\) vectors group-level reg coefs.

      \(\Omega_{\phi}\), \(\Omega_{\pi}\) corr matrix of FG by time; \(\tau_{\phi}\), \(\tau^{\pi}\) scale of FG by time.
    \end{scriptsize}
  \end{block}
\end{frame}


\begin{frame}
  \frametitle{Modeling the probabilities; final details}

  \begin{block}{Comments on priors, implementation}
    \begin{itemize}
      \item Random-walk priors for time-varying intercepts
      \item Regularizing priors predict
        \begin{itemize}
          \item very weak/no effect of mass e.g. \(\mathcal{N}(0, 0.5)\)
          \item very weak/no effect of group-level covariates e.g. \(\mathcal{N}(0, 0.5)\)
          \item very weak/no correlation b/w functional groups e.g. LKJ\((2)\)
        \end{itemize}
      \item Marginalization problem b/c gradient based estimation
    \end{itemize}
  \end{block}
\end{frame}

\begin{frame}
  \frametitle{Parameter estimation and inference}
  \begin{columns}
    \begin{column}{0.45\textwidth}
      \begin{itemize}
        \item \textbf{Bayesian inference}
          \begin{itemize}
            \item intuitive and expressive
            \item regularization/partial pooling
            \item external information
          \end{itemize}
        \item \textbf{Automatic Differentiation Variational Inference (ADVI)}
          \begin{itemize}
            \item when full HMC/MCMC slow
            \item approx Bayesian inference; assumes posterior is Gaussian
            \item true Bayesian posterior
          \end{itemize}
      \end{itemize}
    \end{column}
    \begin{column}{0.55\textwidth}
      \begin{center}
        \includegraphics[height=0.65\textheight,width=\textwidth,keepaspectratio=true]{figure/stan_logo}

        \vspace*{0.05\textheight}

        \LARGE{\textbf{Stan}}
      \end{center}
    \end{column}
  \end{columns}
\end{frame}


\begin{frame}
  \frametitle{Model adequate? Posterior predictive check}
  \begin{center}
    \includegraphics[height=0.8\textheight,width=\textwidth,keepaspectratio=true]{figure/pred_occ_bd}
  \end{center}
\end{frame}

\begin{frame}
  \frametitle{Observation; NALMA}
  \begin{center}
    log-odds of observing a species, given that it is present
    \includegraphics[height=0.775\textheight,width=\textwidth,keepaspectratio=true]{figure/time_observation}
  \end{center}
\end{frame}

\begin{frame}
  \frametitle{Observation; functional group}
  \begin{center}
    log-odds of observing a species, given that it is present
    \includegraphics[height=0.775\textheight,width=\textwidth,keepaspectratio=true]{figure/ecotype_observation}
  \end{center}
\end{frame}

\begin{frame}
  \frametitle{Origination; individual-level}
  \begin{center}
    probability of species originating, given it hasn't originated yet
    \includegraphics[height=0.775\textheight,width=\textwidth,keepaspectratio=true]{figure/ecotype_origin_bd}
  \end{center}
\end{frame}

\begin{frame}
  \frametitle{Origination; group-level}
  \begin{center}
    log-odds of species originating, given it hasn't originated yet
    \includegraphics[height=0.775\textheight,width=\textwidth,keepaspectratio=true]{figure/group_on_origin_bd}
  \end{center}
\end{frame}

\begin{frame}
  \frametitle{Survival; individual-level}
  \begin{center}
    probability of species surviving, given it was present
    \includegraphics[height=0.775\textheight,width=\textwidth,keepaspectratio=true]{figure/ecotype_survival_bd}
  \end{center}
\end{frame}

\begin{frame}
  \frametitle{Survival; group-level}
  \begin{center}
    log-odds of species surviving, given it was present
    \includegraphics[height=0.775\textheight,width=\textwidth,keepaspectratio=true]{figure/group_on_survival_bd}
  \end{center}
\end{frame}

\begin{frame}
  \frametitle{Standing diversity of functional groups through time}
  \begin{center}
    \includegraphics[height=0.8\textheight,width=\textwidth,keepaspectratio=true]{figure/ecotype_diversity}
  \end{center}
\end{frame}

\begin{frame}
  \frametitle{Relative diversity of functional groups through time}
  \begin{center}
    \includegraphics[height=0.9\textheight,width=\textwidth,keepaspectratio=true]{figure/relative_diversity}
  \end{center}
\end{frame}


\begin{frame}
  \begin{block}{Changes to relative diversity between Neogene/Paleogene}
    \begin{itemize}
      \item \alert{increase}
        \begin{itemize}
          \item digitigrade, plantigrade, unguligrade herbivores
          \item fossorial functional groups
          \item plantigrade omnivores
        \end{itemize}
      \item \alert{decrease}
        \begin{itemize}
          \item near total loss of arboreal functional groups
          \item plantigrade, scansorial insectivores
          \item unguligrade omnivores
        \end{itemize}
    \end{itemize}
  \end{block}
\end{frame}

\begin{frame}
  \begin{alertblock}{Conclusions}
    \begin{itemize}
      \item temporal differences in P(observation) much larger than effects of functional group
      \item increases in P(origination) often met with decreases in P(survival), but not 1-to-1
      \item environment estimated to effect origination of functional groups more often than survival
      \item no correlation between functional group origination, survival not accounted for by RW prior
        \begin{itemize}
          \item does not preclude short similarity, just no long term correlation
          \item HMC/MCMC might tweak these results b/c ADVI assumptions
        \end{itemize}
    \end{itemize}
  \end{alertblock}
\end{frame}


\begin{frame}
  \frametitle{Acknowledgements}
  \begin{columns}
    \begin{column}{0.5\textwidth}
      \begin{itemize}
        \item UC Berkeley
          \begin{itemize}
            \item \textbf{Seth Finnegan}, \\Adiel Klompmaker, \\Emily Orzechowski, \\Larry Taylor, \\Sara Kahanamoku, \\Josh Zimmt
          \end{itemize}
        \item UChicago
          \begin{itemize}
            \item \textbf{Kenneth D. Angielczyk}, \\\textbf{Michael J. Foote}, \\P. David Polly, \\Richard H. Ree, \\Graham Slater
          \end{itemize}
      \end{itemize}
    \end{column}
    \begin{column}{0.5\textwidth}
      \begin{center}
        \includegraphics[height=0.2\textheight,width=0.5\textwidth,keepaspectratio=true]{figure/twitter} 

        @PeterDSmits
      \end{center}
      \vspace*{0.05\textheight}
      \begin{center}
        \includegraphics[height=0.4\textheight,width=\textwidth,keepaspectratio=true]{figure/paleodb}
      \end{center}
    \end{column}
  \end{columns}
\end{frame}

\end{document}
