\begin{abstract}

  The set of species in a region changes over time as new species enter through speciation or immigration and as species leave the system through extinction and extirpation. How a regional species pool changes over time is the product of many processes acting at multiple levels of organization. Changes in the functional composition of a regional species pool are changes that occur across all local communities drawn from that species pool. While a speciess presence in a local community is due to the availability of the necessary biotic-biotic or biotic-abiotic interactions that enable coexistence, a species' presence in a regional species pool just requires that at least one local community has that set of necessary interactions. The goal of this analysis is to understand when, and possibly for what reasons, mammal ecotypes are enriched or depleted relative to their average diversity. Here, I analyze the diversity history of North American mammals ecotypes for most of the Cenozoic (the last 65 million years). This analysis frames mammal diversity in terms of both their means of interacting with the biotic and abiotic environment (i.e. functional group or ecotype) as well as their regional and global environmental context. 


\end{abstract}
