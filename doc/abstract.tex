\begin{quotation}
  All the world's a stage, And all the men and women merely players; They have their exits and their entrances\dots
\end{quotation}
\attrib{Shakespeare, \textit{As You Like It}, Act II, Scene VII}

\begin{abstract}

  The set of species in a region changes over time as new species enter through speciation or immigration and as species leave the system through extinction and extirpation. How a regional species pool changes over time is the product of many processes acting across multiple levels of organization. Changes in to the functional composition of a regional species pool are changes that occur across all local communities drawn from that species pool. While a species being present in a local community is due to the availability of the necessary biotic-biotic or biotic-abiotic interactions that enable coexistence, a species being present in a regional species pool just requires the possibility that is at least one local community that has that set of necessary interactions. The goal of this analysis is to understand when, and possibly for what reasons, are mammal ecotypes enriched or depleted relative to their average diversity. Here, I analyze the diversity history of North American mammals ecotypes for most of the Cenozoic (the last 65 million years). This analysis frames mammal diversity in terms of both their means of interacting with both the biotic and abiotic environment (i.e. functional group or ecotype) as well as their regional and global environmental context. Using two hierarchical Bayesian hidden Markov models of diversity, I find that changes to mammal diversity is driven more by the influx of new species than by selective extinction. I also find that the only ecotypes which experience a near constant increase in diversity over time are digitigrade and unguligrade herbivores, while arboreal ecotypes become increasingly rare and in many cases disappear entirely from the species pool over the Cenozoic. Additionally, I find that global temperature is only associated with the origination of some mammal ecotypes but, in almost all cases, does not affect the extinction of mammal ecotypes. The clear and direct translation of research question to statistical model allows for precise and better contextualized results. By taking into account more of the complexity surrounding and contributing to species diversity and the diversification process, the idiosyncrasies of ecotype diversification histories are more clearly contextualized than before.


\end{abstract}
