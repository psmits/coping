\documentclass[12pt,letterpaper]{article}

\usepackage{amsmath, amsthm}
\usepackage{microtype, parskip}
\usepackage[comma,numbers,sort&compress]{natbib}
\usepackage{lineno}
\usepackage{docmute}
\usepackage{caption, subcaption, multirow, morefloats, rotating}
\usepackage{wrapfig}

\frenchspacing

\begin{document}

\section*{Introduction}

% Taxon occurrence as a function of both emergent biological traits and its environmental context?

% fourth-corner type problem
%   presence_{i,t} | individual-level_{i,t}, group-level_{t}
%   assumes effect of covariates constant for all t
% no individual X group covariates
%   instead of group-level covariates included as part of multi-level model
%   keeping in mind group-level covariates depend on t

% after \citep{Smits2015}
%   decrease in extinction risk over time
%     notice the pattern at the paleogene-neogene boundary
%   higher extinction risk in arboreal taxa
%     strong effect
%     what is variation in effect? paleogene vs neogene
%       is the transition/neogene effect so strong to appear constant?
%     driven by the environmental change?

% Vallesian crisis in Europe
%   faunal shift at Paleogene-Neogene boundary
%   do we see this in NA in terms of ecotypes?
% break-point model is the eventual goal

% analysis of mammal genus fossil record for north america
% time is discrete as 2-My bins
%   starting post-K/Pg boundary
%   (ignores quarternary)

% individual-level covariates
%   intercept term, varying by time
%   locomotor type/category
%       arboreal, digitigrade, plantigrade, unguligrade, fossorial, scansorial
%   dietary category
%     carnivore, herbivore, insectivore, omnivore
%   body size
%     rescale of log body mass

% group-level covariates
%   intercept
%   mean of isotope at time t
%     from rescale of isotope
%   interquartile range of isotope at time t
%     from rescale of isotope
%   mean of temperatue estimate at time t
%   interquartile range of temperatue estimate at time t
%   plant community phase following Graham


\end{document}
