\documentclass[12pt,letterpaper]{article}

\usepackage{amsmath, amsthm}
\usepackage{microtype, parskip}
\usepackage[comma,numbers,sort&compress]{natbib}
\usepackage{lineno}
\usepackage{docmute}
\usepackage{caption, subcaption, multirow, morefloats, rotating}
\usepackage{wrapfig}

\frenchspacing

\begin{document}

\section*{Introduction}

% Taxon occurrence as a function of both emergent biological traits and its environmental context?

How do species pools change over time as species are recruited or go extinct? When are ecotypes enriched or depleted? How does global and regional environmental context affect the distribution of species ecotypes (e.g. guilds) in a regional species pool?

A regional species pool is the set of species which form communities in a specific region; local communities are subsets of the regional pool. The composition of a regional species pool changes over time due to speciation, migration, extinction. Local scale processes like resource competition only affect the regional species pool if all communities are affected.

Valentine and Bambach how they presented guilds in paleobiology. Bush and Bambach presented an ecocube to describe what how marine invertebrates partition space and resources \citep{Bush2007,Bambach2007,Bush2011}. Unique combinations represent what possible ecotypes are observable. The distribution of ecocube occupancy is then normally analyzed as raw counts of unique combinations or using ordination methods and the change in disparity over time is estimated \citep{Bush2007,Bambach2007,Bush2011}.

One of the greatest challenges with analyzing species occurrence data is the inherent incompleteness of any sample \citep{Royle2008,Royle2014,Foote1999a,Foote2001,Lloyd2011,Wang2016b}. In the modern, only presences are certain as an absence can be caused by both the species being truly absent or the species never having been sampled \citep{Royle2008,Royle2014}. For paleontological data in the context of this study, the incomplete preservation of fossil communities combined with the incomplete sampling of what fossils there are means that the true times of origination or extinction may not be observed \citep{Foote1999a,Foote2001,Wang2015,Wang2016b}


% hypotheses about the effect of ecotypes
%   difficult to tease out because people use taxonomic grouping, which conflates common ancestry and ecology
% things people have remarked upon in the past
%   cursorial carnivores are a late occurrence (Janis and Wilhelm 93 JME)
%   ungulate legs get longer over Cenozoic but due to climate change (Janis 08)
%   guilds stable in face of turnover EUROPE (jernvall and fortelius 04 amnat)
%   larger body size, greater extinction rate (liow et al 08 pnas)
%   ungulate extinction/turnover at end of eocene (janis 97)
%   browse to graze within herbivores
\citet{Smits2015b} found several systematic differences in mammal species durations associated with various species traits. Omnivorous taxa were found to have, on average, a greater duration than other dietary categories. Additionally, arboreal taxa were found to have a shorter duration than other locomotor categories. 

An unresolved question from \citet{Smits2015b} is whether the greater extinction risk faced by arboreal is constant over time or if there was a change in extinction risk at the Paleogene/Neogene boundary. Specifically, the question is whether the extinction risk arboreal taxa increased in the Neogene, driving the loss of arboreal taxa and average extinction risk of arboreal taxa down. 

There are no observed massive cross-taxonomic turnover events in the North American record, unlike the Neogene record Europe \citep{Alroy2009,Alroy1996a,Eronen2015,Janis1993a,Alroy2000g}.

% if we control for one ecotype axis, what is variation along other axis?
% digitigrade vs unguligrade herbivores
% plantigrade vs digitigrade carnivores
% modernization of ecologies?
%   this is normally talked about in terms of taxonomic groups
%   how does this translate into ecologies?




% effect of climate on diversification process
%   all species respond differently, we want to see if there are any emergent macroecological similarities
% big patterns in isotope records
%   PETM and the Eocene thermal maximum
%   terminal Eocene
%   mid-miocene climatic optimum
%   plio-pleistocene glaciation
%   stable from TE to MMCO
%     decline before, decline after
% what have people said before
%   ungulates get hit at end of Eocene b/c cool (Janis 97)
%   climate is not correlated with mammal diversity or body size (Alroy et al 2000)
%   it is all mountain building (Badgley and Finarelli 13, Eronen et al 15 ProcB, Janis 93)
%     tectonic events drive climate, climate then affects species
%     taxonomic dependence
%   idiosyncratic and variable link between diversity and climate (Figueirido et al 12 PNAS)
%     taxonomic dependence
The effect of climate on diversity and the diversification process has been the focus of considerable research with many analyses favoring diversification being more biologically-mediated than climate-mediated \citep{Alroy1996a,Alroy2000g,Figueirido2012,Clyde1998a}. Scale of analysis makes a big difference in interpretation of results, both temporal and geographic. For example when the mammal fossil record analyzed at small temporal and geographic scales a correlation between diversity and climate are observable \citep{Clyde1998a}. However, when the record is analyzed at the scale of the continent and the Cenozoic there is no correlation with diversity and climate \citep{Alroy2000g}. This results, however, does not go against the idea that there may be short periods of correlation and that this correlation change or reverse direction over time; instead this result means that there is no single direction of correlation between diversity and climate \citep{Figueirido2012}. 

In the case of a fluctuating correlation between diversity and climate it is hard to make the argument of an actual causal link between the two without understanding the ecological differences in mammalian fauna over time; when this analysis is based on diversity or taxonomy alone no mechanisms are possible to infer. After all, taxonomy conflates many potential factors that could affect diversification into a single variable; by separating the effects of shared common ancestry (i.e. phylogeny) from species ecology the subtle differences in the diversification process can be observed \citep{Smits2015b}.

There are many candidate climatic events that may have influenced the distribution of mammal ecotypes regionally, if not globally \citep{Zachos2001,Zachos2008,Janis1993a,Blois2009}. The Paleocene-Eocene Temermal Maximum is associated with species dwarfing and rearrangement of local communities, though regional effects are less known CITATION. The Mid-Miocene climactic optimum is associated with WHAT CITATION. The 

The general cooling throughout the Cenozoic and the development of ice-caps in the Neogene. The Oligo-Miocene boundary. 

One of the most stunning environmental transitions of the Cenozoic in North America was gradual ``opening-up'' of the landscape with the shift from closed or partially forested environments of the Paleogene to the savannah and grasslands environments that characterize the Neogene \citep{Blois2009,Janis1993a,Janis2000,Stromberg2005}.


% events of interest
%   faunal shift at Paleogene-Neogene boundary
%     Oligo-Miocene boundary (Chattian--Aquitanian)
%     no known climatic events
%     from kind of modern to mostly modern species pool
%     (mulitple european events)
%   rise of grasses
%     janis
%     stromberg
%     (fortelius and europe)
%   other climate events? PETM, mid-miocene climatic optimum



\begin{figure}[ht]
  \centering
  \includegraphics[width=\textwidth,height=0.8\textheight,keepaspectratio=true]{figure/paleo_fourth_corner}
  \caption[Conceptual diagram of the paleontological fourth-courner problem]{Conceptual diagram of the paleontological fourth corner problem. The observed presence matrix (orange) is the empirical presence/absence pattern for all species for all time points; this matrix is an incomplete observation of the ``true'' presence/absence pattern (purple). The estimated true presence matrix is modeled as a function of both environmental factors over time (blue) and multiple species traits (green). Additionally, the affect of environmental factors on species traits are also modeled as traits are expected to mediate the effects of a species environmental context. This diagram is based partially on material presented in \citet{Brown2014c} and \citet{Warton2015a}.}
  \label{fig:concept_fourth_corner}
\end{figure}

Fourth-corner modeling an approach to explaining the patterns of either species abundance or presence/absence as a product of species traits, environmental factors, and the interaction between traits and environment CITATION. In modern ecological studies, what is being modeled is species occurrences at localities distributed across a region CITATION. In this study, what is being modeled is the pattern of species occurrence over time for most of the Cenozoic in North America (Fig. \ref{fig:concept_fourth_corner}). These two approaches, modern and palentologicial, are different views of the same three-dimensional pattern: species at localities over time. The temporal limitations of modern ecological studies and difficulties with uneven spatial occurrences of fossils in paleontological studies means that these approaches are complimentary but reveal different patterns of how species are distributed in time and space.

%The major species trait included in this study is species ecotype, defined as the combination of species dietary category (e.g. carnivore) and locomotor category (e.g. arboreal). This trait is a descriptor of a species' ecological guild, similar to the unique combinations of the ecocube used in analysis of marine invertebrates \citep{Bush2007,Bambach2007,Bush2011}. Importantly, the probability of a species ecotype being present was modeled as a function of environmental factors (Fig. \ref{fig:concept_fourth_corner}). Species mass was also included as a species trait, but primarily just to control for that effect on species occurrence. 
%
%
%The environmental factors included in this study include estimates of global tempreature and the changing floral groups present in North America across the Cenozoic. Why are these factors important? What are the hypotheses associated with environmental context?



% why consider phylogeny?
%   may or may not be an issue in community ecology/assembly/whatever the fuck
%     that's whatever
%   but i'm doing stuff over time
%     so we actually progress along the tree
%     much better change of phylogeny actually mattering
%       clade replacement in NA (carnivores, herbivores, etc)
%   so i'm just thinking of it as how it affects occurrence
%     GLMM approach is super effective for this purpose
%   note that i'm not actually modeling the diversification process
%     that's what a true J-S model (or PyRate) do
%     i'm just accounting for phylogenetic autocorrelation in species occurrence


% goals of this analysis
%   questions being addressed
%   how these questions are answered

Ultimately, the goal of this analysis are to understand when are unique ecotypes enriched or depleted in the North American mammal regional species pool and how changes in ecotypic diversity are related to changes in species' environmental context.


\end{document}
