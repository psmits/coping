\documentclass[12pt,letterpaper]{article}

\usepackage{amsmath, amsthm}
\usepackage{microtype, parskip}
\usepackage[comma,numbers,sort&compress]{natbib}
\usepackage{lineno}
\usepackage{docmute}
\usepackage{caption, subcaption, multirow, morefloats, rotating}
\usepackage{wrapfig}

\frenchspacing

\begin{document}

% what influences changes in probabilty of any species being of a given ecotype?
%   given both other aspects of niche and contemporary environment
%   in many ways, this study builds on the ideas of the bambach cube concept
%     key is to actually develop a statistical, not just verbal, model
%   demographic shift along one (compound) aspect of niche
%     what underlies changes in demography between time units?


% hierarchical structure of analysis
%   unit of analysis: species occurrence i
%     with ecotype \in K
%     in time bin t
%   covariates
%     individual-level (properties of species occurrence i)
%     group-level (properties of time bin t)


% response
%   ecotype (demographic descriptor)
%     locomotor type/category
%       arboreal, digitigrade, plantigrade, unguligrade, fossorial, scansorial
%     idea is to capture an aspect of terrestrial guilds
%       need to look at Janis and Van Valkenburg writings
%       alternative approach than, say, bambach cube
%     compound trait as digiti, planti, and unguli are ``ground dwelling'' forms


% individual-level covariates
%   intercept term if doing varying intercepts model
%   dietary category
%     carnivore, herbivore, insectivore, omnivore
%     varying intercepts?
%       drawn from normal mean 0, estimated sd given seperate intercept term
%   body size
%     rescale of log body mass
%   regional environment
%     need terrestrial biomes/biogeographic units
%     varying intercepts by biome (people have an easier time interpreting them)
%       drawn from normal mean 0, estimated sd given seperate intercept term
%       assign based on majority of taxon's occurrences
%   eventually: \uppercase{\textbf{vary with time?}}

% group-level covariates
%   global temperature
%     rescale of delta in degrees 
%   currently/forever: \uppercase{\textbf{don't vary with time?}}


% too many variables? when have i reached too many?
%   approximate number of regression parameters is the above \uppercase{times K}
%     not counting all the additional hierarchical parameters
%   horseshoe priors on everything except intercept?




% after \citep{Smits2015}
%   decrease in extinction risk over time
%     notice the pattern at the paleogene-neogene boundary
%   higher extinction risk in arboreal taxa
%     strong effect
%     what is variation in effect? paleogene vs neogene
%       is the transition/neogene effect so strong to appear constant?
%     driven by the environmental change?

% see Vallesian crisis in Europe
%   faunal shift at Paleogene-Neogene boundary
%   do we see this in NA in terms of ecotypes?

% How does the ratio of ecotypes change over the Cenozoic?
%   decreases (loss) in ecotypes indicate selective pressure AGAINST
%   increases in ecotypes indicate selective pressure FOR
%   accounting for general stochasticity, of course

% Hierarchical Bayesian modeling approach in order to help control for
%   sample size
%   weak effects

\end{document}
