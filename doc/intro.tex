\documentclass[12pt,letterpaper]{article}

\usepackage{amsmath, amsthm}
\usepackage{microtype, parskip}
\usepackage[comma,numbers,sort&compress]{natbib}
\usepackage{lineno}
\usepackage{docmute}
\usepackage{caption, subcaption, multirow, morefloats, rotating}
\usepackage{wrapfig}

\frenchspacing

\begin{document}

% after \citep{Smits2015}
%   decrease in extinction risk over time
%     notice the pattern at the paleogene-neogene boundary
%   higher extinction risk in arboreal taxa
%     strong effect
%     what is variation in effect? paleogene vs neogene
%       is the transition/neogene effect so strong to appear constant?
%     driven by the environmental change?

% see Vallesian crisis in Europe
%   faunal shift at Paleogene-Neogene boundary
%   do we see this in NA in terms of ecotypes?

% How does the ratio of ecotypes change over the Cenozoic?
%   decreases (loss) in ecotypes indicate selective pressure AGAINST
%   increases in ecotypes indicate selective pressure FOR

% Hierarchical Bayesian modeling approach in order to help control for
%   sample size
%   weak effects

\end{document}
