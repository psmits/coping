\documentclass[12pt,letterpaper]{article}

\usepackage{amsmath, amsthm}
\usepackage{microtype, parskip}
\usepackage[comma,numbers,sort&compress]{natbib}
\usepackage{lineno}
\usepackage{docmute}
\usepackage{caption, subcaption, multirow, morefloats, rotating}
\usepackage{wrapfig}

\frenchspacing

\begin{document}

\section*{Introduction}

% Taxon occurrence as a function of both emergent biological traits and its environmental context?

How do species pools change over time as species are recruited or go extinct? When are ecotypes enriched or depleted? How does global and regional environmental context affect the distribution of species ecotypes (e.g. guilds) in a regional species pool?

A regional species pool is the set of species which form communities in a specific region; local communities are subsets of the regional pool. The composition of a regional species pool changes over time due to speciation, migration, extinction. Local scale processes like resource competition only affect the regional species pool if all communities are affected.

Valentine and Bambach how they presented guilds in paleobiology. Bush and Bambach presented an ecocube to describe what how marine invertebrates partition space and resources CITATION. Unique combinations represent what possible ecotypes are observable. The distribution of ecocube occupancy is then normally analyzed using ordination methods and the change in disparity over is estimated CITATION.

Fourth-corner modeling is concerned with explaining either species abundance or presence/absence as a product of species traits, environmental factors, and the interaction between these factors. In modern ecological studies, the matrix being modeled is of species occurrence at localities distributed in region. In this study, the matrix being modeled is of species occurrence in temporal bins across the Cenozoic in North America. These dimensions are all axes of the same three dimensional occurrence matrix: species by locality by time.

One of the greatest challenges with analyzing species occurrence data is the inherent incompleteness of any sample CITATION. In the modern, only presences are certain as an absence can be caused by both the species being truly absent or the species never having been sampled CITATION. For paleontological data in the context of this study, the incomplete preservation of fossil communities combined with the incomplete sampling of what fossils there are means that the true times of origination or extinction may not be observed CITATION.

\citet{Smits2015} found several systematic differences in mammal species durations associated with various species traits. Omnivorous taxa were found to have, on average, a greater duration than other dietary categories. Additionally, arboreal taxa were found to have a shorter duration than other locomotor categories. 

% system
%   Cenozoic mammals of NA
%   lots of things to think about
%     what specific hypotheses are of interest? 
%     why those ones?
% after \citep{Smits2015}
%   decrease in extinction risk over time
%     notice the pattern at the paleogene-neogene boundary
%   higher extinction risk in arboreal taxa; 2 possible situations
%     this effect is constant for all time
%     Paleogene-Neogene transition
%       neutral effect of arboreality during Paleogene
%       strong selection against arboreality during Neogene
%       Neogene effect is stronger than Paleogene effect
%       means that overall mean effect is closer to Neogene
%   former implies no appreciable demographic differences over Cenozoic
%   latter implies a difference in demography between Paleogene and Neogene
% events of interest
%   faunal shift at Paleogene-Neogene boundary
%     Oligo-Miocene boundary (Chattian--Aquitanian)
%     no known climatic events
%     from kind of modern to mostly modern species pool
%     (mulitple european events)
%   rise of grasses
%     janis
%     stromberg
%     (fortelius and europe)
%   other climate events? PETM, mid-miocene climatic optimum


% analysis of mammal genus fossil record for north america
%   time is discrete as 2-My bins
%   starting post-K/Pg boundary
%   (ignores quarternary)

% individual-level covariates
%   intercept term, varying by time
%   locomotor type/category
%       arboreal, digitigrade, plantigrade, unguligrade, fossorial, scansorial
%   dietary category
%     carnivore, herbivore, insectivore, omnivore
%   body size
%     rescale of log body mass

% group-level covariates
%   intercept
%   mean of temperatue estimate at time t
%   interquartile range of temperatue estimate at time t
%   plant community phase following Graham

% compare model assuming perfect sampling to model allowing for imperfect sampling
% imperfect sampling
%   two-state, discrete time hidden markov model with absorbing state
%     observation model for Royle and Dorazio
%     used a lot in paleobiology
%       Liow
%   a continuous-time analogue would be PyRate
%     strong assumptions about probability of observing over species duration
% issues surrounding model complexity
%   including covariates adds a lot of complexity
%   only doing approximate Bayesian inference (ADVI)
%   see methods section for longer discussion

% why consider phylogeny?
%   may or may not be an issue in community ecology/assembly/whatever the fuck
%     that's whatever
%   but i'm doing stuff over time
%     so we actually progress along the tree
%     much better change of phylogeny actually mattering
%       clade replacement in NA (carnivores, herbivores, etc)
%   so i'm just thinking of it as how it affects occurrence
%     GLMM approach is super effective for this purpose
%   note that i'm not actually modeling the diversification process
%     that's what a true J-S model (or PyRate) do
%     i'm just accounting for phylogenetic autocorrelation in species occurrence


% goals of this analysis
%   questions being addressed
%   how these questions are answered

\end{document}
