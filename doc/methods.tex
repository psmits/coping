\documentclass[12pt,letterpaper]{article}

\usepackage{amsmath, amsthm}
\usepackage{microtype, parskip}
\usepackage[comma,numbers,sort&compress]{natbib}
\usepackage{lineno}
\usepackage{docmute}
\usepackage{caption, subcaption, multirow, morefloats, rotating}
\usepackage{wrapfig}

\frenchspacing

\begin{document}
\section*{Materials and Methods}

\subsection*{Taxon occurrence information}

%https://paleobiodb.org/data1.2/occs/list.csv?datainfo&rowcount&base_name=Mammalia&taxon_reso=species&interval=Maastrichtian,Gelasian&cc=NOA&show=class,genus,ecospace,loc,strat,stratext,lith,acconly

Taxon occurrence information was downloaded from the Paleobiology Database on May 1st 2016. Occurrences were restricted to all Mammalia between the Maastrichtian and Gelasian stages and  the North American continent. These occurrences were then further programatically restricted from the Danian through the Gelasian. The taxonomic and stratigraphic metadata for each occurrence, such as life habit and dietary category, was also downloaded. From this base downloaded some taxa were further excluded by their taxonomic and life habit information. Specifically, both volant and aquatic mammals along with the orders and families associated with those life habits were excluded. 

% taxonomic stuff from Smits 2015

Life habit and dietary categories of all occurrences were adjusted following TABLE. 

Life habit is greatly expanded compared to previous studies (\citep{Smits2015}).

Instead of a single ``ground dwelling'' category, species initially classified as ``ground dwelling'' were then further divided by foot posture following \citet{Carrano1997}. Otherwise, ground dwelling is overwhelming and uninterestingly, demographically. The classification by footposture gives more precise information about the structure of the regional species pool wrt environmental context and time.

Species mass data was sourced from the Paleobiology Database, SMITH, TOMIYA, ETC., ETC.
% mass information is an amalgam of
%   PBDB mass value
%   PBDB measurement + regression
%   NOW
%   other papers (Tomiya, Raia, Smith, etc. -- see \citet{Smits2015})


%\subsection*{Supertree inference}
% species level phylogeny
%   also folding in genus level trees by just choosing a random species in that genus
%   Halliday tree
%   Tomiya tree
%   Raia et al tree
%   Binida-Edmonds supertree
%   taxonomy 
%     PBDB
%     Janis books
%     Encylopedia of Life via taxize
% just use MRP
% this is an improvement on \citet{Smits2015}

% how am i time scaling the tree?



\subsection*{Model specification}

% model diagram
%   goal is to translate this conceptual diagram into a statistical model



% basic mechanics
%   hidden markov model with an absorbing state
%   also known as a capture-mark-recapture model
%   constrain transition matrix to only 2 probabilities (existing, not existing)

This model is simply a hidden markov model with an absorbing state. The notation follows the Jolly-Seber capture-mark-recapture model presented in Royle and Dorazio (book).

The observed state of a species at a given time is a function of both its ``true'' state (present or absent) and the probability observation. 



% three important probabilities: observation, existing, not existing
%   all of these vary with time
%   each of these is modeled as a type of logistic regression
%   observation
%     species: mass
%   existing, not existing 
%     species: mass, diet X life
%     group: global temperature (mean, range), NA plant phase
The underlying hidden Markov model used in this study has three characteristic probabilities: probability \(p\) of observing a species given that it is present, probability \(\phi\) of a species surviving from one time to another, and probability \(\pi\) of a species first appearing. Because the questions being addressed in this study are concerned only with species presence, \(\phi\) and \(\pi\) are constrained to be equal and referred to as \(\theta\): the probability that a species is present at time \(t\) given whatever state that species was in at time \(t - 1\).

The probability of observation a species \(p\) was modeled as a logistic regression with a time-varying intercept and species mass as a predictor. The effect of species mass on probability of observation was assumed linear and constant over time and hypothesized to have a positive relationship; this is reflected both in the structure of the model and choice of parameters. The parameters associated with this aspect of the model are described in Table \ref{tab:obs_param}.

\begin{table}
  \centering
  \begin{tabular}{c l}
    Parameter & explanation \\
    \hline
    \(p\) & probability of observation \\
    \(\alpha_{0}\) & average observation probability \\
    \(\alpha_{1}\) & change in average observation probability per change in standard deviation of mass \\
    \(r_{t}\) & difference in average observation probability at time \(t\) \\
    \(\sigma\) & standard deviation of distribution of \(r_{t}\) \\
  \end{tabular}
  \caption{Observation parameters}
  \label{tab:obs_param}
\end{table}


The probability of species occurrence was modeled as a two-level logistic regression: species-level variation and group-level variation. The species-level of the model has a single predictor and an intercept that varies by ecotype. The varying-intercepts are themselves modeled as functions of the group-level covariates. Species mass was assumed to have a quadratic relationship with occurrence probability; this was based on the known distribution of mammal body masses where species with an intermediate mass are more common than either small or large bodied extremes. This assumption was also reflected in the choice of priors for those regression coefficients (see below). The parameters associated with this part of the model are presented in Table \ref{tab:pres_param}. 

\begin{table}
  \centering
  \begin{tabular}{c l l}
    Parameter & dimensions & explanation \\
    \hline
    \(\theta\) & \(N \times T - 1\) & probability of species presence \\
    \(a\) & \(T - 1 \times D\) & effect of ecotype on log-odds of occurrence \\
    \(b_{1}\) & 1 & effect of species mass on log-odds of occurrence \\
    \(b_{2}\) & 1 & effect of species mass, squared, on log-odds of occurrence \\
    \(\gamma\) & \(U \times D\) & matrix of group-level regression coefficients \\
    \(\Sigma\) & \(D \times D\) & covariance matrix between species-level effects over time \\
    \(\Omega\) & \(D \times D\) & correlation matrix between species-level effects over time \\
    \(\tau\) & \(D\) & vector of standard deviations of distributions of species-level effects over time\\
  \end{tabular}
  \caption{Presence parameters}
  \label{tab:pres_param}
\end{table}


Following recommendations from CITATION, CITATION, CITATION all parameters not modeled elsewhere were given weakly informative priors. Weakly informative means that priors do not necessarily encode actual prior information but instead help regularize or weakly constrain posterior estimates. These priors have a concentrated probability density around and near zero; this has the effect of tempering our estimates and help prevent overfitting the model to the data CITATION. 

\begin{equation}
  \begin{aligned}
    y_{i, t} &\sim \text{Bernoulli}(p_{i, t} z_{i, t}) \\
    p_{i, t} &= \text{logit}^{-1}(\alpha_{0} + \alpha_{1} m_{i} + r_{t}) \\ 
    r_{t} &\sim \mathcal{N}(0, \sigma) \\
    z_{i, t} &\sim \text{Bernoulli}(\theta_{i, t}) \\
    \theta_{i, t} &= \text{logit}^{-1}(a_{t, j[i]} + b_{1} m_{i} + b_{2} m_{i}^{2}) \\
    a &\sim \text{MVN}(u \gamma, \Sigma) \\
    \Sigma &= \text{diag}(tau) \Omega \text{diag}(tau) \\
    \alpha_{0} &\sim \mathcal{N}(0, 1) \\
    \alpha_{1} &\sim \mathcal{N}(1, 1) \\
    \sigma &\sim \mathcal{N}^{+}(1) \\
    b_{1} &\sim \mathcal{N}(0, 1) \\
    b_{1} &\sim \mathcal{N}(-1, 1) \\
    \gamma &\sim \mathcal{N}(0, 1) \\
    \tau &\sim \mathcal{N}^{+}(1) \\
    \Omega &\sim \text{LKJ}(2) \\
  \end{aligned}
  \label{eq:full}
\end{equation}


% two models; two fitting techniques
%   one without sampling (implied presence model)
%     range-through occurrence; assumes observation without error
%       model effect of traits on occurrence; not diversification
%       analogous to comparing samples from different sites
%     full stochastic sampling Bayes
%       NUTS HMC for posterior sampling
%   one with sampling (latent discrete model)
%     only observed presences; estimate probability of observing a true occurrence
%     model is very complex --> individual level effects are difficult to estimate
%     variational bayes (like EM algorithm)
%       MAP estimate
%       samples from posterior approximated as uncorrelated Gaussians?
%       ADVI
%       NEED TO READ MORE
% why does this matter?
%   issues surrounding the fossil record and sampling are real
%     the w/ sampling model takes \uppercase{forever} to fully sample the posterior
%     finding the mode is a lot faster
%   ADVI is only approximate; ignores skewness, kertosis present in true posterior
%     can't make confident inference for specific questions
%     maybe estimate how different by using one model and both methods
%   so, fit the w/o sampling model with both NUTS and ADVI
%   compare parameter estimates for covariate effects
%     HS nuts vs HS advi vs LD advi

% appendix information
%   writing the model in stan
%   latent discrete parameter
%     lots of help from stan-users

\subsection*{Posterior inference and model adequacy}

% Stan
%   version 2.9.0
%   marginalize over latent discrete parameters
%     keeping in mind that species HAVE to range through
%     sum of the log probability for all possible configutations
%     along with the sampling probability given those combinations as well
%     see code for implementation? 
%       it is kind of an annoying amount of math to write up

% full bayes for implied presence model
%   not too hard
% ADVI for latent discrete model
%   large/complicated model 
%     too slow for full, stochastic sampling based bayesian inference
%   variational inference simplifies the problem but does fit an actual joint probability

% for posterior simulations, just start from t = 1 and roll forward
%   one-ahead estimate/leave-one-out cross-validation 
% mean number of occurrences in the whole N x T block
% edge effects -- exclude the first and last from analysis?


% repository information
%   all code for cleaning, modeling, plotting.
%   all data for the trees and the unique PBDB download call
% should I make it more readable/automatic?


\end{document}
