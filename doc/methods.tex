\documentclass[12pt,letterpaper]{article}

\usepackage{amsmath, amsthm}
\usepackage{microtype, parskip}
\usepackage[comma,numbers,sort&compress]{natbib}
\usepackage{lineno}
\usepackage{docmute}
\usepackage{caption, subcaption, multirow, morefloats, rotating}
\usepackage{wrapfig}

\frenchspacing

\begin{document}
\section{Materials and Methods}

\subsection{Taxon occurrence information}

% same basic occurrence information from \citep{Smits2015}
% diet and locomotion based on PBDB
% mass information is an amalgam of
%   PBDB mass value
%   PBDB measurement + regression
%   NOW
%   other papers (Tomiya, Raia, Smith, etc. -- see \citet{Smits2015})

\subsection{Supertree inference}

% genus level phylogeny
%   Halliday tree
%   Tomiya tree
%   Raia et al tree
%   Binida-Edmonds supertree
%   taxonomy 
%     PBDB
%     Janis books
%     Encylopedia of Life via taxize


\subsection{Model specification}

% hierarchical model following Royle and Dozario
%   hierarchical just means giving a hyperparameter it's own prior distribution

Define \(y\) as a \(N \times T\) matrix of the observed 0/1 occurrence of species \(i\) at time \(t\), and \(z\) as a \(N \times T\) matrix of the ``true'' 0/1 occurrence of species \(i\) at time \(t\). The observation presence of a taxon \(y_{i,t}\) is modeled as following a Bernoulli distribution defined
\begin{equation}
  y_{i,t} \sim \text{Bernoulli}(\rho_{t} z_{i,t})
  \label{eq:latent}
\end{equation}
where \(\rho\) is a length \(T\) vector of the probability that a species is observed at time \(t\) given that it is present at time \(t\) (\(\rho_{t} = \text{Pr}(y_{\_t} = 1 | x_{\_t} = 1)\)).


Observation probability \(\rho_{t}\) is modeled as Bernoulli distributed variable with probability equal to the inverse logit transformed length \(T\) vector \(\rho^{\prime}\) (Eq. \ref{eq:sample}). The elements of \(\rho^{\prime}\) are modeled as exchangeable draws from a Normal distribution with location \(\rho^{''}\) and scale \(\sigma_{\rho^{\prime}}\); these are given weakly informative Normal and half-Cauchy priors, respectively. % citation?
\begin{equation}
  \begin{aligned}
    \rho_{t} &\sim \text{Bernoulli}\left(\text{logit}^{-1}(\rho^{\prime}_{t})\right) \\
    \rho^{\prime}_{t} &\sim \mathcal{N}(\rho^{''}, \sigma_{\rho^{\prime}}) \\
    \rho^{''} &\sim \mathcal{N}(0, 1) \\
    \sigma_{\rho^{\prime}} &\sim \text{C}^{+}(1) \\
  \end{aligned}
  \label{eq:sample}
\end{equation}


The ``true'' presence/absence states of taxa \(z\) are modeled as a logistic regression (Eq. \ref{eq:presence}) where the probability of being present (\(z = 1\)) is a function of individual-level and group-level effects. The intercept of this regression \(\alpha\) is a length \(T\) vector which varies by time \(t\) (Eq. \ref{eq:time_effect}). Individual-level covariates are a \(N \times D\) matrix \(x\) whose regression coefficients are the length \(D\) column vector \(\beta\). These regression coefficients are given independent weakly informative Normally distributed priors.
\begin{equation}
  \begin{aligned}
    z_{i,t} &\sim \text{Bernoulli}\left(\text{logit}^{-1}(\alpha_{t} + x_{i} \beta)\right) \\
    \beta &\sim \mathcal{N}(0, 1) \\
  \end{aligned}
  \label{eq:presence}
\end{equation}


The value of \(\alpha_{t}\) is itself defined as a regression (Eq. \ref{eq:time_effect}). The intercept of the this group-level regression is defined as \(\mu\). The effect of plant regime \(p\) is modeled as the length \(P\) vector \(\phi\) whose elemnts are draws from a Normal distribution with a location of 0 and scale of \(\sigma_{\phi}\). The group-level covariates corresponding to global climate values of interest is defined as the \(T \times U\) matrix \(u\) where \(\gamma\) is a length \(U\) column vector of regression coefficients. Finally \(\sigma_{\mu}\) is positive real value. \(\mu\), the elements of \(\gamma\), \(\sigma_{\mu}\), and \(\sigma_{\phi}\) are all given weakly informative independent priors (Eq. \ref{eq:time_effect}).
\begin{equation}
  \begin{aligned}
    \alpha_{t} &\sim \mathcal{N}(\mu + \phi_{p[t]} + u_{t} \gamma, \sigma_{\mu}) \\
    \phi_{p} &\sim \mathcal{N}(0, \sigma_{\phi}) \\
    \mu &\sim \mathcal{N}(0, 5) \\
    \gamma &\sim \mathcal{N}(0, 1) \\
    \sigma_{\mu} &\sim \text{C}^{+}(1) \\
    \sigma_{\phi} &\sim \text{C}^{+}(1) \\
  \end{aligned}
  \label{eq:time_effect}
\end{equation}

\end{document}
