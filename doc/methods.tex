\documentclass[12pt,letterpaper]{article}

\usepackage{amsmath, amsthm}
\usepackage{microtype, parskip}
\usepackage[comma,numbers,sort&compress]{natbib}
\usepackage{lineno}
\usepackage{docmute}
\usepackage{caption, subcaption, multirow, morefloats, rotating}
\usepackage{wrapfig}

\frenchspacing

\begin{document}
\section{Materials and Methods}

\subsection{Taxon occurrence information}

\subsection{Model specification}

The mean (i.e. expected) body mass of any origination cohort is inherently known with error. I chose to model this error, or the distribution of log body mass of a given origination cohort, as a skew-normal distribution. This makes sense given that the observed distribution of mammal log body masses is highly skewed CITATION.

The probability density function of the skew-normal distribution is defined

The expectation of the skew-normal distribution 

\(y_{i}\) is the log body mass of species \(i\) which originates at some time \(t\), where \(i = 1, 2, \dots, N\) and \(t = 1, 2, \dots, T\) and \(N\) is total sample size and \(T\) is total number of origination cohorts.

\end{document}
