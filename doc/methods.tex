\documentclass[12pt,letterpaper]{article}

\usepackage{amsmath, amsthm}
\usepackage{microtype, parskip}
\usepackage[comma,numbers,sort&compress]{natbib}
\usepackage{lineno}
\usepackage{docmute}
\usepackage{caption, subcaption, multirow, morefloats, rotating}
\usepackage{wrapfig}

\frenchspacing

\begin{document}
\section*{Materials and Methods}

\subsection*{Taxon occurrence information}

%https://paleobiodb.org/data1.2/occs/list.csv?datainfo&rowcount&base_name=Mammalia&taxon_reso=species&interval=Maastrichtian,Gelasian&cc=NOA&show=class,genus,ecospace,strat,stratext,lith,acconly

Taxon occurrence information was downloaded from the Paleobiology Database on May 1st 2016. Occurrences were restricted to all Mammalia between the Maastrichtian and Gelasian stages and  the North American continent. These occurrences were then further programatically restricted from the Danian through the Gelasian. The taxonomic and stratigraphic metadata for each occurrence, such as life habit and dietary category, was also downloaded. From this base downloaded some taxa were further excluded by their taxonomic and life habit information. Specifically, both volant and aquatic mammals along with the orders and families associated with those life habits were excluded. 

% taxonomic stuff from Smits 2015

Life habit and dietary categories of all occurrences were adjusted following TABLE. 

Life habit is greatly expanded compared to previous studies (\citep{Smits2015}).

Instead of a single ``ground dwelling'' category, species initially classified as ``ground dwelling'' were then further divided by foot posture following \citet{Carrano1997}. Otherwise, ground dwelling is overwhelming and uninterestingly, demographically. The classification by footposture gives more precise information about the structure of the regional species pool wrt environmental context and time.

Species mass data was sourced from the Paleobiology Database, SMITH, TOMIYA, ETC., ETC.
% mass information is an amalgam of
%   PBDB mass value
%   PBDB measurement + regression
%   NOW
%   other papers (Tomiya, Raia, Smith, etc. -- see \citet{Smits2015})


%\subsection*{Supertree inference}
% species level phylogeny
%   also folding in genus level trees by just choosing a random species in that genus
%   Halliday tree
%   Tomiya tree
%   Raia et al tree
%   Binida-Edmonds supertree
%   taxonomy 
%     PBDB
%     Janis books
%     Encylopedia of Life via taxize
% just use MRP
% this is an improvement on \citet{Smits2015}

% how am i time scaling the tree?



\subsection*{Model specification}

% base model
%   core is a 2-state markov model
%   contrain 0 --> 1 and 1 --> 1 to be equal
%     probability only for being present
%   absorbing condition for extinction (1 --> 0)
%   this follows Royle and Dorazio
% occurrence probability
%   hierarchical logistic regression

\(y^{r}\) is a \(N \times D\) matrix of implied (range-through) occurrences.

\(x\) is a \(N \times D\) matrix of individual-level covariates. \(x_{i}\) is a row vector.

\(\beta\) is a \(D \times T\) matrix of regression coefficients. \(\beta_{t}\) is a column vector.

\(u\) is a \(J \times T\) matrix of group-level covariates. \(u_{t}\) is a row vector.

\(\gamma\) is a \(J \times D\) matrix of regression coefficients.

\(\Sigma\) is a \(D \times D\) covariance matrix.

\begin{equation}
  \begin{aligned}
    y^{r}_{i,t} &\sim \text{Bernoulli}(\theta_{i,t}) \\
    \theta_{i, t} &= \text{logit}^{-1}(y^{r}_{i, t - 1} (x_{i} \beta_{t}) + \prod_{k = 1}^{k = t - 1}(1 - y^{r}_{i, t = k}) (x_{i} \beta_{t})) \\
    \beta_{t} &\sim \text{MVN}(u_{t} \gamma, \Sigma). \\
  \end{aligned}
  \label{<++>}
\end{equation}

The product term ensures that loss is an absorbing state (i.e. no de-extinction).


Horseshoe priors for the regression coefficients with shared shrinkage for each group-level covariate. Remember, each group-level covariate affects each individual-level covariate (including the intercept terms).

\(\lambda\) is a length \(J\) vector of scales.

\(\phi\) is a \(J \times D\) matrix of scales.

\begin{equation}
  \gamma_{j, d} &\sim \mathcal{N}(0, \lambda_{j}\phi_{j, d}) \\
  \lambda_{j} &\sim \text{C}^{+}(1) \\
  \phi_{j, d} &\sim \text{C}^{+}(1) \\
  \label{eq:horseshoe}
\end{equation}


Decompose \(\Sigma\) into a vector of scale terms \(\tau\) and a correlation matrix \(\Omega\): \(\Sigma = \text{diag}(\tau) \Omega \text{diag}(\tau)\). \(\tau\) is a length \(D\) vector of the between time point scale for each individual-level covariate and \(\Omega\) is a \(D \times D\) correlation matrix; both of these are given weakly informative priors. \(\tau\) is given a half-Cauchy prior and \(\Omega\) is given an LKJ prior following STAN MANUAL; these are defined
\begin{equation}
  \begin{aligned}
    \tau &\sim \text{C}^{+}(1) \\
    \Omega &\sim \text{LKJ}_{cor}(2). \\
  \end{aligned}
  \label{<++>}
\end{equation}



% model with imperfect observation
%   make previous model the hidden layer in a 2-state hidden markov model
%   follows Royle and Dozario
\(y^{o}\) is a \(N \times T\) matrix of observed occurrences.

\(z\) is a \(N \times T\) matrix of ``true'' occurrences.

\(p\) is a length \(T\) vector of observation probabilities.

\begin{equation}
  \begin{aligned}
    y^{o}_{i,t} &\sim \text{Bernoulli}(z_{i,t} p_{t}) \\
    z_{i,t} &\sim \text{Bernoulli}(\theta_{i,t}) \\
    \theta_{i, t} &= z_{i, t - 1} (x_{i} \beta_{t}) + \prod_{k = 1}^{k = t - 1}(1 - z_{i, t = k}) (x_{i} \beta_{t}) \\
  \end{aligned}
  \label{<++>}
\end{equation}


% two models; two fitting techniques
%   one without sampling (implied presence model)
%     range-through occurrence; assumes observation without error
%       model effect of traits on occurrence; not diversification
%       analogous to comparing samples from different sites
%     full stochastic sampling Bayes
%       NUTS HMC for posterior sampling
%   one with sampling (latent discrete model)
%     only observed presences; estimate probability of observing a true occurrence
%     model is very complex --> individual level effects are difficult to estimate
%     variational bayes (like EM algorithm)
%       MAP estimate
%       samples from posterior approximated as uncorrelated Gaussians?
%       ADVI
%       NEED TO READ MORE
% why does this matter?
%   issues surrounding the fossil record and sampling are real
%     the w/ sampling model takes \uppercase{forever} to fully sample the posterior
%     finding the mode is a lot faster
%   ADVI is only approximate; ignores skewness, kertosis present in true posterior
%     can't make confident inference for specific questions
%     maybe estimate how different by using one model and both methods
%   so, fit the w/o sampling model with both NUTS and ADVI
%   compare parameter estimates for covariate effects
%     HS nuts vs HS advi vs LD advi

% appendix information
%   writing the model in stan
%   latent discrete parameter
%     lots of help from stan-users


\subsection*{Posterior inference and model adequacy}

% Stan
%   version 2.9.0
%   marginalize over latent discrete parameters
%     keeping in mind that species HAVE to range through
%     sum of the log probability for all possible configutations
%     along with the sampling probability given those combinations as well
%     see code for implementation? 
%       it is kind of an annoying amount of math to write up

% full bayes for implied presence model
%   not too hard
% ADVI for latent discrete model
%   large/complicated model 
%     too slow for full, stochastic sampling based bayesian inference
%   variational inference simplifies the problem but does fit an actual joint probability

% for posterior simulations, just start from t = 1 and roll forward
%   one-ahead estimate/leave-one-out cross-validation 
% mean number of occurrences in the whole N x T block
% edge effects -- exclude the first and last from analysis?


% repository information
%   all code for cleaning, modeling, plotting.
%   all data for the trees and the unique PBDB download call
% should I make it more readable/automatic?


\end{document}
