\documentclass[12pt,letterpaper]{article}

\usepackage{amsmath, amsthm}
\usepackage{microtype, parskip}
\usepackage[comma,numbers,sort&compress]{natbib}
\usepackage{lineno}
\usepackage{docmute}
\usepackage{caption, subcaption, multirow, morefloats, rotating}
\usepackage{wrapfig}

\frenchspacing

\begin{document}
\section*{Materials and Methods}

\subsection*{Taxon occurrence information}

%https://paleobiodb.org/data1.2/occs/list.csv?datainfo&rowcount&base_name=Mammalia&taxon_reso=species&interval=Maastrichtian,Gelasian&cc=NOA&show=class,genus,ecospace,strat,stratext,lith,acconly

Taxon occurrence information was downloaded from the Paleobiology Database on May 1st 2016. Occurrences were restricted to all Mammalia between the Maastrichtian and Gelasian stages and  the North American continent. These occurrences were then further programatically restricted from the Danian through the Gelasian. The taxonomic and stratigraphic metadata for each occurrence, such as life habit and dietary category, was also downloaded. From this base downloaded some taxa were further excluded by their taxonomic and life habit information. Specifically, both volant and aquatic mammals along with the orders and families associated with those life habits were excluded. 

% taxonomic stuff from Smits 2015

Life habit and dietary categories of all occurrences were adjusted following TABLE. 

Species mass data was sourced from the Paleobiology Database, SMITH, TOMIYA, ETC., ETC.
% mass information is an amalgam of
%   PBDB mass value
%   PBDB measurement + regression
%   NOW
%   other papers (Tomiya, Raia, Smith, etc. -- see \citet{Smits2015})


%\subsection*{Supertree inference}
% genus level phylogeny
%   Halliday tree
%   Tomiya tree
%   Raia et al tree
%   Binida-Edmonds supertree
%   taxonomy 
%     PBDB
%     Janis books
%     Encylopedia of Life via taxize


\subsection*{Model specification}

% hierarchical just means giving a hyperparameter it's own prior distribution
% model follows a lot of Royle and Dozario


% two models
%   one without sampling (implied presence model)
%   one with sampling (latent discrete model)

\subsection*{Posterior inference and model adequacy}

% Stan
%   version 2.9.0
%   marginalize over latent discrete parameters
%     keeping in mind that species HAVE to range through
%     sum of the log probability for all possible configutations
%     along with the sampling probability given those combinations as well
%     see code for implementation? 
%       it is kind of an annoying amount of math to write up
%
% full bayes for implied presence model
%   not too hard
% ADVI for latent discrete model
%   large/complicated model -- too slow for full, stochastic sampling based bayesian inference
%   variational inference simplifies the problem but does fit a full joint probability


% for posterior simulations, just start from t = 1 and roll forward
%   one-ahead estimate/leave-one-out cross-validation 
%   use AUC of measure of model performance
% estimate ROC curve for posterior estimates from entire dataset




\end{document}
