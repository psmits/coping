\documentclass[12pt,letterpaper]{article}
\usepackage{amsmath, amsthm}

\frenchspacing

\begin{document}
\textbf{Title:} Taxon occurrence as a function of both biological traits and environmental context: the changing North American mammal species pool.

What species, ecologies, or interactions are present in a local community is dependent on many factors principally what species are ``available'' to populate that community. A regional species pool is the set of species from which all local communities in that region are formed. Questions remain, however, of how a regional species pool changes over time and what factors influence the demographic composition of the species pool itself and how that might change over time. For example, are there periods of time were certain trophic and habitat ecologies are especially rare while being common at others? Using a multi-level Bayesian model of species occurrence as a function of both species traits and their temporally varying environmental context, I analyze the demographic composition of the North American mammal species pool and how it has changed over the Cenozoic. The species traits analyzed here are trophic category, habitat interaction/locomotor category, and body size. This analysis of species pool demography specifically models the effects of these species traits as functions of their the environmental context, such as global temperature, regional floral assemblage.


% after \citep{Smits2015}
%   decrease in extinction risk over time
%     notice the pattern at the paleogene-neogene boundary
%   higher extinction risk in arboreal taxa; 2 possible situations
%     this effect is constant for all time
%     Paleogene-Neogene transition
%       neutral effect of arboreality during Paleogene
%       strong selection against arboreality during Neogene
%       Neogene effect is stronger than Paleogene effect
%       means that overall mean effect is closer to Neogene
%   former implies no appreciable demographic differences over Cenozoic
%   latter implies a difference in demography between Paleogene and Neogene
% events of interest
%   faunal shift at Paleogene-Neogene boundary
%     Oligo-Miocene boundary (Chattian--Aquitanian)
%     no known climatic events
%     from kind of modern to mostly modern species pool
%     (mulitple european events)
%   rise of grasses
%     janis
%     stromberg
%     (fortelius and europe)
%   other climate events? PETM, mid-miocene climatic optimum

% analysis of mammal species fossil record for north america



\end{document}
