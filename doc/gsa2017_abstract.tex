\documentclass[12pt,letterpaper]{article}
\usepackage{amsmath, amsthm}

\frenchspacing

\begin{document}
\textbf{Title:} Modeling changes to the functional composition of North American mammal diversity: multi-level dynamics of a regional species pool 

The functional composition of regional diversity reflects the possible ecological interactions observable in any given community. The relative fitness of species functional roles leads do differences in origination and extinction, which in turns leads to changes to functional composition and a new set of observable interactions. Changes to the relative diversity of functional groups over time can indicate the loss or gain of critical habitat or environmental interactions. Understanding when functional groups are enriched or depleted relative to the long-term average is critical for understanding relative conservation importance; a low diversity group that is currently enriched relative to its historical diversity is most likely at less risk than a high diversity group which has less-than-average standing diversity. I develop a multi-level Bayesian model of species occurrence where observation, origination, and extinction are modeled as functions of functional group, global temperature, and North American plant community phase. Mammal functional group is considered the interaction between species dietary and locomotor categories. I find that digitigrade and unguligrade herbivores experience an increase in relative diversity over the Cenozoic; fossorial functional groups experience a similar but much smaller relative increase. In contrast, all arboreal functional groups decline in relative diversity and in some cases become absent from the species pool. Additionally, I find the environmental covarites such as temperature have no effect on instantaneous extinction probability, and instead are only associated with differences in origination probability.

\end{document}

