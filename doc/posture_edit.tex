
\begin{center}
  \begin{longtable}{ l l l }
    \caption[Posture assignment based on taxonomy]{Ankle posture assignment as based on taxonomy. Assignments are based on \citep{Carrano1997}. Taxonomic groups are presented alphabetically and without reference for the nestedness of families in orders. As such, do not infer higher-order structure from this table.} \label{tab:posture} \\

    Order & Family & Stance \\ \hline
    \endfirsthead
  
    \multicolumn{3}{p{\textwidth}}{{ \bfseries \tablename\ \thetable{} -- continued from previous page}} \\
    \hline Order & Family & Stance \\ \hline
    \endhead
      
    \hline \multicolumn{3}{p{\textwidth}}{{Continued on next page}} \\ \hline
    \endfoot
  
    \hline \hline
    \endlastfoot
  
    & Ailuridae & plantigrade \\ 
    & Allomyidae & plantigrade \\ 
    & Amphicyonidae & plantigrade \\ 
    & Amphilemuridae & plantigrade \\ 
    & Anthracotheriidae & digitigrade \\ 
    & Antilocapridae & unguligrade \\ 
    & Apheliscidae & plantigrade \\ 
    & Aplodontidae & plantigrade \\ 
    & Apternodontidae & scansorial \\ 
    & Arctocyonidae & unguligrade \\ 
    & Barbourofelidae & digitigrade \\ 
    & Barylambdidae & plantigrade \\ 
    & Bovidae & unguligrade \\ 
    & Camelidae & unguligrade \\ 
    & Canidae & digitigrade \\ 
    & Cervidae & unguligrade \\ 
    & Cimolodontidae & scansorial \\ 
    & Coryphodontidae & plantigrade \\ 
    & Cricetidae & plantigrade \\ 
    & Cylindrodontidae & plantigrade \\ 
    & Cyriacotheriidae & plantigrade \\ 
    & Dichobunidae & unguligrade \\ 
    Dinocerata &  & unguligrade \\ 
    & Dipodidae & digitigrade \\ 
    & Elephantidae & digitigrade \\ 
    & Entelodontidae & unguligrade \\ 
    & Eomyidae & plantigrade \\ 
    & Erethizontidae & plantigrade \\ 
    & Erinaceidae & plantigrade \\ 
    & Esthonychidae & plantigrade \\ 
    & Eutypomyidae & plantigrade \\ 
    & Felidae & digitigrade \\ 
    & Florentiamyidae & plantigrade \\ 
    & Gelocidae & unguligrade \\ 
    & Geolabididae & plantigrade \\ 
    & Glyptodontidae & plantigrade \\ 
    & Gomphotheriidae & unguligrade \\ 
    & Hapalodectidae & plantigrade \\ 
    & Heteromyidae & digitigrade \\ 
    & Hyaenidae & digitigrade \\ 
    & Hyaenodontidae & digitigrade \\ 
    & Hypertragulidae & unguligrade \\ 
    & Ischyromyidae & plantigrade \\ 
    & Jimomyidae & plantigrade \\ 
    Lagomorpha &  & digitigrade \\ 
    & Leptictidae & plantigrade \\ 
    & Leptochoeridae & unguligrade \\ 
    & Leptomerycidae & unguligrade \\ 
    & Mammutidae & unguligrade \\ 
    & Megalonychidae & plantigrade \\ 
    & Megatheriidae & plantigrade \\ 
    & Mephitidae & plantigrade \\ 
    & Merycoidodontidae & digitigrade \\ 
    Mesonychia &  & unguligrade \\ 
    & Mesonychidae & digitigrade \\ 
    & Micropternodontidae & plantigrade \\ 
    & Mixodectidae & plantigrade \\ 
    & Moschidae & unguligrade \\ 
    & Muridae & plantigrade \\ 
    & Mustelidae & plantigrade \\ 
    & Mylagaulidae & fossorial \\ 
    & Mylodontidae & plantigrade \\ 
    & Nimravidae & digitigrade \\ 
    & Nothrotheriidae & plantigrade \\ 
    Notoungulata &  & unguligrade \\ 
    & Oromerycidae & unguligrade \\ 
    & Oxyaenidae & digitigrade \\ 
    & Palaeomerycidae & unguligrade \\ 
    & Palaeoryctidae & plantigrade \\ 
    & Pampatheriidae & plantigrade \\ 
    & Pantolambdidae & plantigrade \\ 
    & Periptychidae & digitigrade \\ 
    Perissodactyla &  & unguligrade \\ 
    & Phenacodontidae & unguligrade \\ 
    Primates &  & plantigrade \\ 
    & Procyonidae & plantigrade \\ 
    & Proscalopidae & plantigrade \\ 
    & Protoceratidae & unguligrade \\ 
    & Reithroparamyidae & plantigrade \\ 
    & Sciuravidae & plantigrade \\ 
    & Sciuridae & plantigrade \\ 
    & Simimyidae & plantigrade \\ 
    & Soricidae & plantigrade \\ 
    & Suidae & digitigrade \\ 
    & Talpidae & fossorial \\ 
    & Tayassuidae & unguligrade \\ 
    & Tenrecidae & plantigrade \\ 
    & Titanoideidae & plantigrade \\ 
    & Ursidae & plantigrade \\ 
    & Viverravidae & plantigrade \\ 
    & Zapodidae & plantigrade \\ 
    \hline
  \end{longtable}
\end{center}
