\documentclass[12pt,letterpaper]{article}

\usepackage{amsmath, amsthm}
\usepackage{microtype, parskip}
\usepackage[comma,numbers,sort&compress]{natbib}
\usepackage{lineno}
\usepackage{docmute}
\usepackage{caption, subcaption, multirow, morefloats, rotating}
\usepackage{wrapfig}
\usepackage{attrib}

\frenchspacing

\begin{document}

\section*{Discussion}

%Over time both the stage and the actors change, but not at the same rate. As the actors change, do their parts remain?
Both species pools and environmental context change over time, though not at the same rate. Local communities, who's species are drawn from this pool, have ``roles'' in their communities as their interact with a host of biotic and abiotic interactors. For higher level ecological characterizations like ecotypes and guilds, these roles are broadly defined and not about specific interactions but the genre of interactions covered by that grouping. 

It has been observed that the diversity of an ecotype or guild can be stable over millions of years despite constant species turnover \citep{Jernvall2004,Slater2015c} CITATIONS. This implies that the size and scope of the role of an ecotype or guild is preserved even as the individual interactors change.

What is the pace of environmental change in North America over the Cenozoic? Is it a constant process or a pulsed one?




Comparison of the pure-presence model to the birth-death model support the conclusion that regional species pool dynamics cannot simply be described by a single probability of occurrence and is instead the product of both origination and extinction. Additionally, changes to ecotypic composition of the North American regional species pool are driven primarily by variation in origination rates. This aspect of how regional species pool diversity is shaped is not observable from studies of the Modern CITATION.

The time scale available with paleontological data is much greater than that obtainable from modern ecological studies, even long running observations CITATION. Specifically, the temporal scale of paleontological data allows for the complete turnover of a species pool to be observed, something that is impossible in ``real time.'' However, paleontological data is very limited in its spatial resolution, so the analysis of how the ecotypic diversity local communities change over time and how that is also the product of larger scale regional turnover remains unanswered.

Phylogenetic comparative community ecology and phylogenetic comparative biogeography also discusses how the macroevolutionary processes helps structure an observed community, though it is not necessarily phrased that way. However, that community did not form in isolation but it the result of many factors interacting over time including incumbency, competition, limiting similarity, etc.



How do the results line up with previous observations and hypotheses discussed in the introduction? 

Extinction rate for the entire regional species pool through time is highly variable and demonstrates a saw-toothed pattern around an apparently constant mean. While a constant mean extinction rate is consistent with previous observation \citep{Alroy1996a,Alroy2000g}, the degree to which extinction rate is actually variable may not have been equally appreciated. What is most consistent with previous observations \citep{Alroy1996a,Alroy2000g}, however, is that diversity seems to be most structured by origination than extinction.

Plant phase always affects plantigrade ecotypes. Does it affect all plantigrade taxa?

Temperature affects very little in general with a few major exceptions: origination probabilities of digitigrade carnivores, and digitigrade and unguligrade herbivores. Why?

Arboreal taxa disappear over the Cenozoic, with massive disappearance by the Paleogene-Neogene barrier. This is consistent with one of the possible explanations presented: Paleogene-Neogene are different and while the earliest Cenozoic may have been neutral wrt arboreal taxa, they disappeared quickly which may account for their higher extinction risk.

Digitigrade carnivores have a relatively stable diversity history through the Cenozoic and could be characterized as varying around a constant mean diversity. This result is consistent with similar observations in \citet{Slater2015c,Silvestro2015b}.

Both digitigrade and unguligrade herbivores increase in diversity over the Cenozoic. The increase of these cursorial forms is consistent with the gradual opening up of the North American landscape CITATION.



What these results support is a gradual change to the ecotypic diversity of the regional species pool for the Cenozoic.

The rapidity of Cenozoic environmental change is worth discussing. If change is rapid, ecotypic composition of species pool does not seem to track environmental change. If change is gradual then there is the possibility that changes to ecotypic composition may be tracking environmental change.



The effects of phylogeny on origination and extinction are not directly considered in this analysis. While a birth-death process approximates the speciation-extinction process \citep{Silvestro2014a} this is not same as considering how the similarity between closely related species may affect estimates for the effects of species traits or response to environmental on both origination and extinction \citep{Smits2015b,Harnik2014}. One of the principle barriers to the inclusion of the effect of phylogeny in either the pure-presence or birth-death models is computational; with over 1000 tips, the calculation of the scale parameter defining phylogenetic effect would be very slow and further increase the already slow computation time necessary for both the marginalization of the discrete occurrence histories and data augmentation already included in both models.

The effect of species mass on either occurrence or origination and extinction was not allowed to vary by ecotype even though there may be difference amoungst those ecotypes CITATION. The primary reason for this modeling choice was this studies focus on ecotypic differences in occurrence, or origination and extinction. Allowing the effect of this covariate to vary by ecotype, time, and environmental factors would increase the overall complexity of the model, something that may not be necessary because the covariate is not the focus of this study. Instead, this covariate was included in order to control for its possible underlying effects CITATION. Additionally, body size was allowed to have a second-order polynomial form and no higher order polynomials were considered; this was done \uppercase{because}

The only covariate allowed to affect sampling probability is mass and only as a linear predictor. Other covariates, such as the environmental factors considered here, may have affected the underlying preservation process that limits sampling probability. It should be noted that in other similar studies that use a hidden birth-death model to handle simultaneous estimation of sampling, origination, and extinction have not considered the possible effects of covariates, both species traits and environmental factors, on sampling CITATION.


An ideal system would have the temporal scale of the fossil record combined with the spatial scale afforded in studies of extant systems.






\end{document}
