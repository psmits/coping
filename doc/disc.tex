\documentclass[12pt,letterpaper]{article}

\usepackage{amsmath, amsthm}
\usepackage{microtype, parskip}
\usepackage[comma,numbers,sort&compress]{natbib}
\usepackage{lineno}
\usepackage{docmute}
\usepackage{caption, subcaption, multirow, morefloats, rotating}
\usepackage{wrapfig}
\usepackage{attrib}

\frenchspacing

\begin{document}

\section*{Discussion}


Both the composition of a species pool and its environmental context change over time, though not necessarily at the same rate or concurrently. Local communities, whose species are drawn from the regional species pool, have ``roles'' in their communities defined by their interactions with a host of biotic and abiotic interactors (i.e. a species' niche). For higher level ecological characterizations like ecotypes and guilds, these roles are broad and not defined by specific interactions but by the genre of interactions species within that grouping participate in. The diversity of species within an functional group can be stable over millions of years despite constant species turnover \citep{Jernvall2004,Slater2015c,Valkenburgh1999}. This implies that the size and scope of the role of an ecotype or guild in local communities, and the regional species pool as a whole, is preserved even as the individual interactors change. This also implies that the structure of regional species pools can be constant over time despite a constantly changing set of ``players.'' There is even evidence that functional groups are at least partially self-organizing and truly emergent \citep{Scheffer2006a}.

The goal of this study has been to understand how macroevolutionary processes are affected by species ecology as well as environmental context, and how these interactions shape macroecological patterns such as regional functional diversity that we observe in the fossil record and the present.


The results of the analysis presented here add considerable nuance to our understanding of mammal macroevolution and macroecology over the Cenozoic. 

This analysis reveals how functional groups respond differently to environmental factors through changes to origination probability but not survival probability.

I focus on functional diversity because of its importance to macroecological patterns McGill CITATION. By analyzing mammals species in the context of their functional ecologies, the complexity of species response to environmental change can be better elucidated. Additionally, by analyzing all of them simultaneously these results are all directly comparable as they are estimated relative to each other.

% synthesis of results

Over the entity of the studied period, each functional group demonstrates its own origination and survival history with no evidence of any longterm cross-correlations. Of course, lack of correlation over the Cenozoic does not preclude similarities in response to individual events; it is just that these correlations are ephemeral and indicate that there is no one relationship between climate and diversity.

Not always the same functional group associated. That's why breaks down over entire cenozoic. individual events aren't series. can affect moments, but not time. sudden shifts vs long term patterns. Respond sometimes, but not others; thus no correlation for Cenozoic; doesn't preclude process association.

Instead, there are individual periods in time that are characterized by similar changes to absolute or relative diversity. Additionally, it is not always the same functional groups that appear to experience changes in abundance at these time points.


The environmental covariates are found to effect the origination probability of some functional groups, but only effect the survival probability of relatively few groups. 

Plant phases: biggest difference is higher origination probability in the Eocene-Miocene phase than the Paleocene-Eocene phase (13 of 18 with P>0.85). Lower origination probability in the Paleocene-Eocene than the Miocene-Pleistocene (10 of 18 with P>0.85). Higher origination in the Eocene-Miocene than Miocene-Pleistocene (8 of 18 P>0.85) exception that 1 of 18 P>0.85 has lower.

There is no evidence that the estimate of global temperature used in this study is not estimated to be a strong predictor of survival probability. Additionally, there is very little evidence (3 of 18 for min 1 phase comparison) of survival probability being different between two plant phases. 

Temperature is estimated to be a predictor of origination probability for many functional groups which either decrease or increase in diversity over the Cenozoic. For example, arboreal and digitigrade herbivores have close opposite diversity histories, but there are similarities in origination probability for which temperature is estimated to be a good predictor. The contrast between these two groups appears in their survival histories; arboreal herbivores have flat survival probabilities for the Cenozoic while digitigrade herbivores have peak in survival approximately 33 Mya.

The result that temperature does not affect the survival probability of most functional groups is consistent with previous analysis of mammal diversity \citep{Alroy2000g}. The result that temperature affects origination probability, on the other hand, is in strong contrast to the results \citet{Alroy2000g}. An important difference between the anlayses presented here and that of \citet{Alroy2000g} is I am considering the effect of temperature on the probability of a species originating, assuming it hasn't originated yet while \citet{Alroy2000g} analyzes the correlation between the first differences of the origination and extinction rates with an oxygen isotope curve \citep{Zachos2001}. Origination or extinction rates have very different properties than the origination probabilities estimated here brought upon by the difference both in definition and units. Origination probability is the expected probability that a species that has never been present and is not present at time \(t\) will be present at time \(t + 1\); origination probability is defined for a single species. In contrast, per capita rates are defined (for origination) as the expected number of new species to have originated between time \(t\) and \(t + 1\) given the total number of species present at time \(t\); per capita rates are defined for the standing diversity. It is also important to note that even though there is an edge effect at the last time interval that causes an increase in the occurrence and origination probabilities of some functional groups (Fig. \ref{fig:eco_origin}. However, it is still possible that the finding that temperature has an effect on origination may simply be because as time approaches the present the number of species which have originated increases and not because of climatic forcing of origination. 

All environmental factors are found to affect the occurrence and origination probabilities for most, but not all, mammal ecotypes (Fig. \ref{fig:group_origin_bd}). In contrast, the environmental factors probably do not affect differences in ecotype survival probability (Fig. \ref{fig:group_surv_bd}). The focus in previous research on temperature and major climatic or geological events without other measures of environmental context may have led to confusion in discussions of how the ``environment'' affects mammal diversity and diversification \citep{Alroy2000g,Figueirido2012}. The environment or climate are more than just global or regional temperature, it is also the set of all possible biotic and abiotic interactions that can be experienced by a member of the species pool. By including more descriptors of species' environmental context than simple an estimate of global temperature a more complete ``picture'' of the diversification process is inferred.


Mass affects observation and origination probability, but not survival probability.




There are three major time units during the Cenozoic important points in time stick out: the Wasatchian, Orellan/Whitneyan, and the Barstovian. The Wasatchian and Barstovian NALMA mark the two major peaks in mammal diversity as estimated by my model, while the Orellan/Whitneyan are estimated to be time lowest diversity during the Cenozoic. Each of these time units mark changes in absolute and relative diversity; for example, the Wasatchian marks peak relative diversity of arboreal carnivores and scansorial herbivores increase in relative diversity after this time point. Here I discuss which functional groups are associated with important changes to the regional species pool at these three periods of time as well as their environmental context.

The Wasatchian

% standing
- peak in all aboreal, peak in digitigrade carnivores, expansion digitigrade herbivores after, peak plantigrade carnivore, peak plantigrade herbivores, high plantigrade insectivores, high scansorial carnivores

% relative
- Changes at 55: peak arboreal carnivore, low fossorial herbivore, expansion scansorial herbivores after


Orellan and Whitneyan is lowest diversity, but is a moment of change in diversity. Marks the loss of some, marks the expansion of others. This also marks a peak in survival for some groups. Are these the groups that expand in the Neogene? Six functional groups are estimated to have peaks in survival probability at these points. Eocene/Oligocene boundary

% standing
- loss/bottom out of all arboreal groups, dip unguligrade herbivore, mark expansion of fossorial groups + plantigrade omnivores, drop off of plantigrade insectivore, scansorial insectivore, scansorial omnivore

% relative
- Changes as 33: increase digitigrade carnivores, digitigrade herbivores, fossorial herbivores, fossorial insectivores, start of increase scansorial omnivores; decrease scansorial insectivores, loss arboreal insectivore


The Barstovian Mid-Miocene Climatic Optimum

% standing
- peak unguligrade herbivore, peak digitigrade carnivore, peak digitigrade herbivore, peak fossorial insectivore

% relative
- Changes at 15: increase arboreal insectivore, plantigrade omnivores, decrease digitigrade herbivores, loss unguligrade omnivore, loss plantigrade insectivore.




Mammal species are short lived with average duration being only slightly more than one NALMA. This short a duration means that observation probability is very high for most of the Cenozoic, with it being greater than 0.50 for the entire Cenozoic while also being greater than 0.80 for most of the Cenozoic. Time is associated with greater variation in observation probability than functional group. Scansorial insectivores are estimated to have a substantially lower observation probability than the other functional groups.










\subsection*{Conclusions}

Is origination driven by ecological opportunity while survival is driven by differences in species-level fitness?

These results add a considerable degree of nuance to the narrative of changes to North American diversity being gradual. My results support the conclusions that functional diversity is shaped more by changes to origination than extinction and that major changes to total diversification rate can be attributed to increases in origination of only some ecotypes. There are a number of interesting estimated ecotype diversity patterns. While arboreal ecotypes are diverse in the Paleogene, by the Neogene all arboreal ecotypes dramatically decreased in diversity and became either rare or absent from the regional species pool. The other ecotypes that decrease in diversity over the Cenozoic are plantigrade and scansorial insectivores and scansorial omnivores. The only ecotypes that demonstrate a sustained pattern of increasing diversity are digitigrade and unguligrade herbivores. When the environmental covariates analyzed here are inferred to affect the diversification of an ecotype, this effect is virtually always on origination and not survival. This analysis provides a much more complete picture of North American mammal diversity and diversification, specifically the dynamics of the ecotypic composition of that diversity. By increasing the complexity of analysis while precisely translating research questions into a statistical model, the context of the results is much better understood. Future studies of diversity and diversification should incorporate as much information as possible into their analyses in order to better understand or at least contextualize the complex processes underlying that diversity.





\end{document}
