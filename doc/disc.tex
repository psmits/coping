\documentclass[12pt,letterpaper]{article}

\usepackage{amsmath, amsthm}
\usepackage{microtype, parskip}
\usepackage[comma,numbers,sort&compress]{natbib}
\usepackage{lineno}
\usepackage{docmute}
\usepackage{caption, subcaption, multirow, morefloats, rotating}
\usepackage{wrapfig}
\usepackage{attrib}

\frenchspacing

\begin{document}

\section*{Discussion}


%The composition of a species pool as well as its environmental context change over time, though not necessarily at the same rate or concurrently. Local communities, whose species are drawn from the regional species pool, have ``roles'' in their communities defined by their interactions with a host of biotic and abiotic interactors (i.e. a species' niche). For higher level ecological characterizations like ecotypes and guilds, these roles are broad and not defined by specific interactions but by the genre of interactions species within that grouping participate in. The diversity of species within an functional group can be stable over millions of years despite constant species turnover \citep{Jernvall2004,Slater2015c,Valkenburgh1999}. This implies that the size and scope of the role of an ecotype or guild in local communities, and the regional species pool as a whole, is preserved even as the individual interactors change. This also implies that the structure of regional species pools can be constant over time despite a constantly changing set of ``players.'' There is even evidence that functional groups are at least partially self-organizing and truly emergent \citep{Scheffer2006a}.

% what are important things from introduction?
%   climate/temperature
%   mass
%   how are my results compare?
% what are unexpected things from the results?
%   what might these mean?
%   what new hypotheses can i propose?
The goal of this study has been to understand how mammal functional diveristy of the North American species pool has changed over time as a result of macroevolutionary and macroecological processes. By analyzing mammals species in the context of their functional ecologies, how species response to environmental change can be better contextualized in terms of environmental interactions, both biotic-biotic and biotic-abiotic. 

One of the principal results from this study is that functional groups have independent responses to changes in their environmental context, and that most functional groups are estimated to respond through changes in environmental context by changes to origination probability and not survival probability. 

Additionally, the functional groups' origination probability time-series are not estimated to be cross-correlated; this is also true for the functional groups' survival probability time-series. Instead, there are individual periods in time that are characterized by similar changes to absolute or relative diversity. Importantly, it is not always the same functional groups that appear to experience changes in absolute or relative diveristy at these time points. The results of the analysis presented here add considerable nuance to our understanding of mammal macroevolution and macroecology over the Cenozoic. 

% what are important things from introduction?
%   climate/temperature
%   mass
%   how are my results compare?
% what are unexpected things from the results?
%   what might these mean?
%   what new hypotheses can i propose?


The environmental covariates are found to effect the origination probability of some functional groups, but only effect the survival probability of relatively few groups. Plant phases: biggest difference is higher origination probability in the Eocene-Miocene phase than the Paleocene-Eocene phase (13 of 18 with P\(>0\).85). Lower origination probability in the Paleocene-Eocene than the Miocene-Pleistocene (10 of 18 with P\(>\)0.85). Higher origination in the Eocene-Miocene than Miocene-Pleistocene (8 of 18 P\(>\)0.85) exception that 1 of 18 P\(>\)0.85 has lower.

There is no evidence that the estimate of global temperature used in this study is not estimated to be a strong predictor of survival probability. Additionally, there is very little evidence (3 of 18 for min 1 phase comparison) of survival probability being different between two plant phases. 

Temperature is estimated to be a predictor of origination probability for many functional groups which either decrease or increase in diversity over the Cenozoic. For example, arboreal and digitigrade herbivores have close opposite diversity histories, but there are similarities in origination probability for which temperature is estimated to be a good predictor. The contrast between these two groups appears in their survival histories; arboreal herbivores have flat survival probabilities for the Cenozoic while digitigrade herbivores have peak in survival approximately 33 Mya.

The result that temperature does not affect the survival probability of most functional groups is consistent with previous analysis of mammal diversity \citep{Alroy2000g}. The result that temperature affects origination probability, on the other hand, is in strong contrast to the results \citet{Alroy2000g}. An important difference between the anlayses presented here and that of \citet{Alroy2000g} is I am considering the effect of temperature on the probability of a species originating, assuming it hasn't originated yet while \citet{Alroy2000g} analyzes the correlation between the first differences of the origination and extinction rates with an oxygen isotope curve \citep{Zachos2001}. Origination or extinction rates have very different properties than the origination probabilities estimated here brought upon by the difference both in definition and units. Origination probability is the expected probability that a species that has never been present and is not present at time \(t\) will be present at time \(t + 1\); origination probability is defined for a single species. In contrast, per capita rates are defined (for origination) as the expected number of new species to have originated between time \(t\) and \(t + 1\) given the total number of species present at time \(t\); per capita rates are defined for the standing diversity. It is also important to note that even though there is an edge effect at the last time interval that causes an increase in the occurrence and origination probabilities of some functional groups (Fig. \ref{fig:eco_origin}. However, it is still possible that the finding that temperature has an effect on origination may simply be because as time approaches the present the number of species which have originated increases and not because of climatic forcing of origination. 

All environmental factors are found to affect the occurrence and origination probabilities for most, but not all, mammal ecotypes (Fig. \ref{fig:group_origin_bd}). In contrast, the environmental factors probably do not affect differences in ecotype survival probability (Fig. \ref{fig:group_surv_bd}). The focus in previous research on temperature and major climatic or geological events without other measures of environmental context may have led to confusion in discussions of how the ``environment'' affects mammal diversity and diversification \citep{Alroy2000g,Figueirido2012}. The environment or climate are more than just global or regional temperature, it is also the set of all possible biotic and abiotic interactions that can be experienced by a member of the species pool. By including more descriptors of species' environmental context than simple an estimate of global temperature a more complete ``picture'' of the diversification process is inferred.



The results of my model suggest that there are three moments of exceptionally high or low mammal species diversity in the Cenozoic: the Wasatchian, Orellan/Whitneyan, and the Barstovian. The Wasatchian and Barstovian NALMA mark the two major peaks in mammal diversity as estimated by my model, while the Orellan/Whitneyan are estimated to be time lowest diversity during the Cenozoic. Each of these time units mark changes in absolute and relative diversity; for example, the Wasatchian marks peak relative diversity of arboreal carnivores and scansorial herbivores increase in relative diversity after this time point. Here I discuss which functional groups are associated with important changes to the regional species pool at these three periods of time as well as their environmental context.


The Wasatchian brackets the PETM and the EECO. In terms of standing diversity, the Wasatchian marks peak diversity of all arboreal functional groups; peak diversity of digitigrade carnivores, plantigrade carnivores, plantigrade herbivores; high diversity of plantigrade insectivores and scansorial carnivores; and the subsequent expansion of diveristy of digitigrade herbivores following this time point. In terms of relative diversity, the Wasatchian marks peak arboreal carnivore, a low amount of fossorial herbivores, and the expansion of scansorial herbivores after this time point. \uppercase{meaning? compare to hypotheses from intro}
In general, these findings are consistent with many of the hypothesized changes to mammal diversity described in \citet{Woodburne2009}.


predict: increase creodons, primates, artiodacyla, perisodacyla. peak browse. drop insectivores. replace phenocodonids and plesiafapids with terrestrial herbivores and frugivores. turnover rodents, primates, pholidotans. euprimates, hypercarnivores, artio, persio. decrease functional diversity after.



There are three major time units during the Cenozoic important points in time stick out: the Wasatchian, Orellan/Whitneyan, and the Barstovian. The Wasatchian and Barstovian NALMA mark the two major peaks in mammal diversity as estimated by my model, while the Orellan/Whitneyan are estimated to be time lowest diversity during the Cenozoic. Each of these time units mark changes in absolute and relative diversity; for example, the Wasatchian marks peak relative diversity of arboreal carnivores and scansorial herbivores increase in relative diversity after this time point. Here I discuss which functional groups are associated with important changes to the regional species pool at these three periods of time as well as their environmental context.



The Wasatchian
% standing
- peak in all aboreal, peak in digitigrade carnivores, expansion digitigrade herbivores after, peak plantigrade carnivore, peak plantigrade herbivores, high plantigrade insectivores, high scansorial carnivores

Orellan and Whitneyan is lowest diversity, and is a major moment in changes to functional diversity. Marks the loss of some, marks the expansion of others. This also marks a peak in survival probability for six functional groups WHICH. These time units also mark the Eocene/Oligocene boundary. In terms of standing diversity, these time units mark the complete loss or the nadir in diversity of all the arboreal functional groups; a dip in diversity of unguligrade herbivores, a drop off in diversity of plantigrade insectivres, scansorial insetivores, and scansorial omnivores; and marks the expansion fossorial functional groups as well as plantigrade omnivores. In terms of relative diveristy, the Orellan and Whitneyan are associated with an increase in digitigrade carnivores, digitigrade herbivores, fossorial herbivores and fossorial insectivores; the beginning of an increase in scansorial omnivores; a decrease in diveristy of scansorial insectivores; and the loss of arboreal insectivores as a meaningful component of relative diversity. \uppercase{meaning? compare to hypotheses from intro}

The near complete loss of arboreal functional groups from the regional species pool is not specifically hypothesized in previous studies of the Eocene-Oligocene transition, but is none the less predictable given the long standing narrative describing the loss of closed, forested environments from North America. What is remarkable, however, is the simultaneous near complete loss of these groups which is one of the most obvious shared changes to functional diversity. This pattern implies a single, shared mechanism underlying the loss of arboreal diversity.

predict: loss ungulate. tectonism




The third moment of particular interest is the Barstovian which marks the second peak in mammal standing diversity as estimated from my model. This time unit also marks the Mid-Miocene Climatic Optimum, a period of relative warm-th and stability in global temperature compared to the rest of the Neogene. In terms of standing diversity, the Barstovian marks peak diveristy of unguligrade herbivores, digitigrade carnivores, digitigrade herbivores, and fossorial insectivores. In contrast, relative diversity is much more varied in terms of changes to functional diversity. The Barstovian marks an increase in relative diversity of arboreal insetivores, plantigrade omnivores; a decrease in relative diversity of digitigrade herbivores; and the loss of the contribution of unguligrade ominvores and plantigrade insectivores to the species pool. \uppercase{meaning? compare to hypotheses from intro}

predict: increase ungulates, rodents. i hypothesize fossorial, unguligrade, and digitigrade increase. tectonism



% observation
Mammal species are short lived with average duration being only slightly more than one NALMA. This short a duration means that observation probability is very high for most of the Cenozoic, with it being greater than 0.50 for the entire Cenozoic while also being greater than 0.80 for most of the Cenozoic. Time is associated with greater variation in observation probability than functional group. Scansorial insectivores are estimated to have a substantially lower observation probability than the other functional groups.

importantly i'm not estimating the missing diversity, just the range extensions. these are fundamentally different. i think people have a lot of confusion about what preservation rate is versus observation probability. i also only need presence at time, not the amount of presences at that time.



% other species level traits
Mass is estimated with XX percent probability of having a TYPE effect on species observation, XX percent probability of having a TYPE effect on species origination, and an XX percent probability of having a TYPE effect on species survival. This marginal result may point to heterogeneity both in time and across diversity wrt when and to whom mass matters wrt survival. remember that liow used only large bodied mammals; this biases results! tomiya used a restricted subset too. i use ``everything'' which means i might be revealing the complexity in response. Heterogeneity is biology. future analysis might consider heterogeneity over time and across taxa, however that is beyond the scope of this study specifically. again, remember this study is about functional groups not survival per se. i'm smoothing over that.






Things I could have done better: 

allow taxon effect to vary with time; adds a weird amount of complexity and potential unidentifiable moments which make this super fucking hard. also, not central to this study. 

my data is PBDB and apparently I need to apologize about that at every fucking turn; that's ok, just get a fucking life or help improve the data or provide cyphers to improve the data programatically. but what about all these errors and missing taxa, you say. fuck you, if it isn't published and in the PBDB i didn't use it. fight me.

the issue of biggest complaint is the actual functional categories and the ``fear'' that they are all based on taxonomy; of course they are! the ankle posture stuff is and i chose to do that; the fact that diet is that way seems super natural to me. including the taxon category as a independent effect is for the hope to help control for that. 

why didn't i just use pyrate? because fossils aren't in continuous time. pyrate makes too many assumptions and this approach trades assumptions for flexibility. also, i'm like one of 10 people in the world who can actually read the pyrate papers, so give up.

better environmental covarites. show me them. if they don't exist, i can't use them.

species average mass assumed constant with time. one) many species only exist one time unit. two) that type of data does not exist for many species. also unnecessary complication.

what about other continents? while the data is ``out there'' it also isn't. also scrappier than NA for Paleogene which is kind critical for this study.

your orders are paraphyletic. shut up. virtually all species were at one point paraphyletic, if not still are. additionally, these orders were at one point monophyletic, especially at the times of their start. this argument means so little to me.



things I wish i could have done:

spatial context would make this amazing. because then i could be talking about changes to average community and its associated spatial heterogeneity






Mammal species are short lived with average duration being only slightly more than one NALMA. This short a duration means that observation probability is very high for most of the Cenozoic, with it being greater than 0.50 for the entire Cenozoic while also being greater than 0.80 for most of the Cenozoic. Time is associated with greater variation in observation probability than functional group. Scansorial insectivores are estimated to have a substantially lower observation probability than the other functional groups.




\subsection*{Conclusions}

The biggest story of macroecology and changes to functional diversity is that the macroevolutionary processes underpinning these changes are heterogeneous over time and across species. 

What do all these results mean? This is the nuance we've needed to actually understand species response to changes in environmental context. Is origination driven by ecological opportunity while survival is driven by differences in species-level fitness?

how does the regional species pool reorganize? no two functional groups are identical in their patterning, instead there are unique events that may be shared across them. this means that many of our questions have been invalid or unanswerable: you can't test for correlation with n=1. it also doesn't mean that environmental events can't effect evolution; it just means it is heterogeneous and partially random. this is most strongly evinced by the fact that similar changes to environments don't involve changes to the same functional groups each time.


Is origination driven by ecological opportunity while survival is driven by differences in species-level fitness?

These results add a considerable degree of nuance to the narrative of changes to North American diversity being gradual. 
While arboreal ecotypes are diverse in the Paleogene, by the Neogene all arboreal ecotypes dramatically decreased in diversity and became either rare or absent from the regional species pool. 
When the environmental covariates analyzed here are inferred to affect the diversification of an ecotype, this effect is virtually always on origination and not survival. This analysis provides a much more complete picture of North American mammal diversity and diversification, specifically the dynamics of the ecotypic composition of that diversity. By increasing the complexity of analysis while precisely translating research questions into a statistical model, the context of the results is much better understood. Future studies of diversity and diversification should incorporate as much information as possible into their analyses in order to better understand or at least contextualize the complex processes underlying that diversity.





\end{document}
