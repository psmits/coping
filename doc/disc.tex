\documentclass[12pt,letterpaper]{article}

\usepackage{amsmath, amsthm}
\usepackage{microtype, parskip}
\usepackage[comma,numbers,sort&compress]{natbib}
\usepackage{lineno}
\usepackage{docmute}
\usepackage{caption, subcaption, multirow, morefloats, rotating}
\usepackage{wrapfig}
\usepackage{attrib}

\frenchspacing

\begin{document}

\section*{Discussion}

%Over time both the stage and the actors change, but not at the same rate. As the actors change, do their parts remain?
Both the composition of a species pool and its environmental context changes over time, though not necessarily at the same rate or concurrently. Local communities, who's species are drawn from the regional species pool, have ``roles'' in their communities defined by their interactions with a host of biotic and abiotic interactors (i.e. a species' niche). For higher level ecological characterizations like ecotypes and guilds, these roles are broad and not defined by specific interactions but by the genre of interactions species within that grouping participate in. The diversity of species within an ecotype or guild can be stable over millions of years despite constant species turnover \citep{Jernvall2004,Slater2015c,Valkenburgh1999}. This implies that the size and scope of the role of an ecotype or guild in local communities, and the regional species pool as a whole, is preserved even as the individual interactors change. This also implies the structure of regional species pools can be constant over time despite a constantly changing set of ``players.'' This result supports the hypothesis that ecotypes or guilds are at least partially self-organizing and truly emergent \citep{Scheffer2006a}.

Comparison of the posterior predictive check results for the pure-presence and birth-death models supports the conclusion that regional species pool dynamics cannot simply be described by a single occurrence probability and is instead the result of the interplay between origination and extinction. Additionally, changes to the ecotypic composition and diversification rate for the North American regional species pool are driven primarily by variation in origination rates. These aspects of how regional species pool diversity is shaped is not directly observable in studies of the Modern where time scales are short and macroevolutionary dynamics are inferable solely from phylogeny \citep{Fritz2013a}.

Extinction rate for the entire regional species pool through time is highly variable and demonstrates a saw-toothed pattern around an apparently constant mean. While a constant mean extinction rate is consistent with previous observation \citep{Alroy1996a,Alroy2000g}, the degree to which extinction rate is actually variable may not have been equally appreciated. What is most consistent with previous observations \citep{Alroy1996a,Alroy2000g}, however, is that diversity seems to be most structured by changes to origination rather than changes to extinction.

Comparison of the ecotype specific diversity histories supports the conclusion that there were no major, simultaneous changes in diversity between the functional groups of the regional speies pool; instead these results support a more gradual and idiosyncratic shifts in relative ecotypic diversity over time (Fig. \ref{fig:ecotype_diversity}). The closest examples to a sudden increase or decrease of a specific ecotype is the jump is standing diversity of scansorial carnivores and, to a lesser extent, fossorial insectivores at 16 Mya (i.e. the start of the third plant phase). This result may, however, not reflect the dynamics of individual local communities as this is an analysis of the entire North American mammal regional species pool.

Arboreal taxa disappear from the regional species pool over the Cenozoic, with long term decline over the Paleogene leading to the disapperance by start of Neogene \(\sim\)22 Mya. This is consistent with one of the two possible patterns presented here and in \citet{Smits2015b} that would result in arboreal taxa having a greater extinction risk than other ecotypes: the Paleogene and Neogene were different selective regimes and while the earliest Cenozoic may have been neutral wrt arboreal taxa, they disappeared quickly over the Cenozoic which may account for their higher extinction risk. In addition to all arboreal taxa, the diversity of plantigrade and scansorial insectivores decreases with time (Fig. \ref{fig:ecotype_diversity}).

Digitigrade carnivores have a relatively stable diversity history through the Cenozoic and can be characterized as varying around a constant mean diversity. This ecotype has a large amount of overlap with the carnivore guild which has been the focus of much research \citep{Slater2015c,Valkenburgh1999,Pires2015a,Janis1993c}. This result is consistent with some form of ``control'' on the diversity of this ecotype, such as diversity-dependent diversification \citep{Slater2015c,Silvestro2015b,Valkenburgh1999}.

Both digitigrade and unguligrade herbivores increase in diversity over the Cenozoic. The increase of these cursorial forms is consistent with the gradual opening up of the North American landscape \citep{Blois2009,Stromberg2005,Graham2011a}. These herbivore increase in diversity over the Cenozoic which may be indicative of a long-term shift in the interactors associated with those ecotypes leading to increased contribution to the regional species pool. This result may be comparable to the increasing percentage of hypsodont (high-crowned teeth) mammals in the Neogene of Europe being due to an enrichment of hyposodont taxa and not a depletion of non-hypsodont taxa. Smaller scale increases in fossorial herbivore species, and a lesser extent plantigrade herbivores, suggests that the increase of interactors may be associated mostly with the herbivore dietary category with locomotor category tempering that relationship. These results support the conclusion that the increase in digitigrade and unguligrade herbivores is the result of an enrichment of these ecotypes as opposed to being caused by the depletion of other herbivorous ecotypes; this is further supported by the lack of major changes to the number of extinctions of all herbivore ecotypes (Fig. \ref{fig:ecotype_death}).

An association between plant phase and differences in ecotype occurrence or origination-extinction probabilities is interpreted to mean that an ecotype enrichment or depletion is due to to associations between that ecotype and whatever plants are dominate at that time and are thus a contributing factor to the constancy of an ecotype, or the lack there of. Plant phase clearly structures the occurrence and origination probability time series (Fig. \ref{fig:eco_occur}, \ref{fig:eco_origin}). These differences in occurrence or origination translate opaquely to the estimates of diversity and diversification rate; the largest spike in both diversity, diversification rate, and origination rate all correspond to the onset of the last plant phase (Fig. \ref{fig:macro_values}). The clearest example of the diversity of an ecotype increasing at this particular transition is in scansorial carnivores (Fig. \ref{fig:ecotype_diversity}); similar shifts in other ecotypes are much more subtle, as was previously noted for fossorial insectivores. 

Interestingly, all of the ecotypes with sudden changes to diversity at this transition increase in diversity, even if only temporarily. There are two interpretations of these results. A biological interpretation of this result is that, because plant phase associations are only with occurrence or origination probabilities and not survival, these ecotypes were well suited for the newly available mammal-plant interactions due to the increased modernization of their floral context \citep{Graham2011a}. Alternatively, the increase in diversity associated wit the the third plant phase may be caused by the edge effect in origination probability that is artificially inflating the number of origination events (Fig. \ref{fig:eco_origin}). However, the estimated number of origination events does not have a tremonedous spike at this transition, nor is a major increase in the number of origination events sustained (Fig. \ref{fig:ecotype_birth}).

There are fewer, less obvious shifts in diversity surrounding the transition from the first to second, with the following ecotypes having apparent shifts in diversity at 50 My: digitigrade carnivores (down), plantigrade carnivores (down), plantigrade herbivores (up), arboreal omnivores (down), and scansorial omnivores (down). Because plant phase has been found to structure occurrence/origination (Fig. \ref{fig:eco_occur}, \ref{fig:eco_origin}), but not survival (Fig. \ref{fig:eco_survival}. My interpretation of these results is that new species were not entering the system because there were fewer available mammal-plant interactions available for those ecotypes. Instead, these ecotypes were poorly suited for the newly available mammal-plant interactions brought upon by the changing environmental context \citep{Graham2011a}.

The estimated effects of temperature on occurrence and origination-extinction probabilities are similar to those of the plant phases. The occurrence and origination probabilities of many mammal ecotypes have strong relationships with the two temperature covariates (Tables \ref{tab:occur_temp}, \ref{tab:origin_temp}). In most cases, there is a negative association between temperature and probability of occurring or first originating; this means that if temperature decreases, we would then expect the probability of occurring or first originating would increase. Contrastingly, only temperature range are estimated to be good predictors of survival in four mammal ecotypes and only marginally for two of those (Table \ref{tab:surv_temp}). Additionally, in all four of these cases are expected to have positive relationships, meaning that if temperature decreases it is expected that species survival will also decrease.

The comparative size of the effects of plant phase and temperature are approximately equal in importance in the sense that they have similar effect sizes on the ecotypes. The focus in previous research on temperature and major climatic or geological events withough other measures of environmental context may have been a mistake and perhaps led to increasing confusion in discussions of how the ``environment'' affects mammal diversity and diversification. The environment or climate is not just global or regional temperature, it is the set of all possible biotic and abiotic interactions that can be experienced by a member of the species pool. By including more descriptors of species' environmental context a more complete ``picture'' of the diversification process is inferred.


The effect of species mass on either occurrence or origination and extinction was not allowed to vary by ecotype or environmental context. The primary reason for this modeling choice was that this study focuses on ecotypic based differences in either occurrence, or origination and extinction. Allowing the effect of body size to vary by ecotype, time, and environmental factors would increase the overall complexity of the model, something that I felt was not necessary because the overall scope of the study. Instead, body size was included in order to control for its possible underlying effects \citep{McElreath2016}. A control means that if there is variation due to body mass, having a term to ``absorb'' that effect is better than ignoring it which may affect other parameter estimates. Additionally, the effect of body size was allowed to have a second-order polynomial form and no higher order polynomials were considered; this was done because it is hard to conceive of a more complex third- or higher-order relationship between body size and the other parameters. Finally, parametric forms of nonlinearity have not previously been considered, so the simple act of estimating a potential second-order relationship is an opportunity to test more complex hypotheses of the relationship between body size and both macroevolutionary and macroecological processes.

The only covariate allowed to affect sampling probability was mass and only as a linear predictor. Other covariates, such as the environmental factors considered here, could have affected the underlying preservation process that limits sampling probability; their exclusion as covariates of sampling/observation was the product of a few key decisions: model complexity, model interpretability, the scope of this study, and a lack of good hypotheses related to these covariates to warrant their inclusion. %It should be noted that in other similar studies that use a hidden Markov model, like both models in this study, to handle simultaneous estimation of sampling, origination, and extinction have not considered the possible effects of covariates, both species traits and environmental factors, on sampling CITATION.

%The time scale available with paleontological data is much greater than that obtainable from modern ecological studies, even long running observations CITATION. Specifically, the temporal scale of paleontological data allows for the complete turnover of a species pool to be observed, something that is impossible in ``real time.'' However, paleontological data is very limited in its spatial resolution, so the analysis of how the ecotypic diversity local communities change over time and how that is also the product of larger scale regional turnover remains unanswered. Phylogenetic comparative community ecology and phylogenetic comparative biogeography also discusses how the macroevolutionary processes helps structure an observed community, though it is not necessarily phrased that way. However, that community did not form in isolation but it the result of many factors interacting over time including incumbency, competition, limiting similarity, etc.

The potential effects of common ancestry (i.e. phylogeny) on origination and extinction are not directly considered in this analysis. While a birth-death process approximates the speciation-extinction process underlying the phylogeny \citep{Silvestro2014a} this is not same as considering how the similarity between closely related species may affect the estimates of the effects of species traits to environmental factors on both origination and extinction \citep{Smits2015b,Harnik2014}. The inclusion of phylogeny can help disentangle if the functional composition of species diversity is shaped either by many closely related species occurring at the same time or if closely related species are more evenly distributed in time; this is analogous to if species within local communities are clumped or dispersed relative to their relatedness \citep{Webb2002,Kraft2007a,Cavender-Bares2009}. One of the principle barriers to the inclusion of the effect of phylogeny in either the pure-presence or birth-death models is computational; with well over 1000 tips, the calculation of the scale parameter defining the phylogenetic effect would be very slow and further increase the already slow computation time necessary for both the marginalization of the discrete occurrence histories and data augmentation already included in both models.

These results support the conclusion that the relative ecotypic diversity of the North American mammal species pool has changed gradually over time. While there is constant species turnover for the entire Cenozoic, there is little evidence of major cross-ecotype upheaval and sudden reorganization of the functional composition of the regional species pool. The results of this study also support the conclusion that mammal diversification over the Cenozoic is driven primarily by changes to origination rate and not extinction rate. There are a number of interesting estimated ecotype diversity patterns. While arboreal ecotypes are diverse in the Paleogene, by the Neogene all arboreal ecotypes dramatically decreased in diversity and became either rare or absent from the regional species pool. The other ecotypes that decrease in diversity over the Cenozoic are plantigrade and scansorial insectivores and scansorial omnivores. Contrastingly, the only ecotypes that demonstrate a sustained pattern of increasing diversity are digitigrade and unguligrade herbivores. Interestingly, when the environmental covariates analyzed here are inferred to affect the diversification of an ecotype, this effect is virtually always for origination and not survival. This analysis provides a much more complete picture of North American mammal diversity and diversification, specifically the dynamics of the ecotypic composition of that diversity. By increasing the complexity of analysis while precisely translating research questions into a statistical model, the context of the results is much better understood. Future studies of diversity and diversification should incorporate as much information as possible into their analyses in order to better understand or at least contextualize the complex processes underlying that diversity.





\end{document}
