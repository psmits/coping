\documentclass[12pt,letterpaper]{article}

\usepackage{amsmath, amsthm}
\usepackage{microtype, parskip}
\usepackage[comma,numbers,sort&compress]{natbib}
\usepackage{lineno}
\usepackage{docmute}
\usepackage{caption, subcaption, multirow, morefloats, rotating}
\usepackage{wrapfig}
\usepackage{attrib}

\frenchspacing

\begin{document}

\section*{Discussion}

%Over time both the stage and the actors change, but not at the same rate. As the actors change, do their parts remain?
Both the composition of a species pool and its environmental context change over time, though not necessarily at the same rate. Local communities, who's species are drawn from the regional species pool, have ``roles'' in their communities defined by their interactions with a host of biotic and abiotic interactors (i.e. species niche). For higher level ecological characterizations like ecotypes and guilds, these roles are broadly defined and not defined by specific interactions but the genre of interactions that species within that grouping participate in. The diversity of species within an ecotype or guild can be stable over millions of years despite constant species turnover \citep{Jernvall2004,Slater2015c} CITATIONS. This implies that the size and scope of the role of an ecotype or guild is preserved even as the individual interactors change. This also implies the structure of regional species pools can be constant over time despite a constantly changing set of ``players.''

Comparison of the pure-presence model to the birth-death model support the conclusion that regional species pool dynamics cannot simply be described by a single occurrence probability and is instead better modeled as the result of both origination and extinction. Additionally, changes to ecotypic composition of the North American regional species pool are driven primarily by variation in origination rates. These aspects of how regional species pool diversity is shaped is not observable from studies of the Modern CITATION.


Extinction rate for the entire regional species pool through time is highly variable and demonstrates a saw-toothed pattern around an apparently constant mean. While a constant mean extinction rate is consistent with previous observation \citep{Alroy1996a,Alroy2000g}, the degree to which extinction rate is actually variable may not have been equally appreciated. What is most consistent with previous observations \citep{Alroy1996a,Alroy2000g}, however, is that diversity seems to be most structured by origination than extinction.


Arboreal taxa disappear over the Cenozoic, with massive disappearance by the Paleogene-Neogene transition \(\sim\)22 Mya. This is consistent with one of the two possible patterns that would result in arboreal taxa having a greater extinction risk than other ecotypes: Paleogene-Neogene are different and while the earliest Cenozoic may have been neutral wrt arboreal taxa, they disappeared quickly over the Cenozoic which may account for their higher extinction risk.

Digitigrade carnivores have a relatively stable diversity history through the Cenozoic and could be characterized as varying around a constant mean diversity. The ecotypie has a large amount of overlap with the carnivore guild which has been the focus of much research CITATIONS. This result is consistent with some form of ``control'' on the ecotype, such as environmental stability, diversity-dependence, or similar \citet{Slater2015c,Silvestro2015b}.

Both digitigrade and unguligrade herbivores increase in diversity over the Cenozoic. The increase of these cursorial forms is consistent with the gradual opening up of the North American landscape CITATION. Only these herbivore from increase in diversity over the Cenozoic which may indicate a long shift in the interactors associated with those ecotypes leading to increased contribution to the regional species pool. This result may be comparable to the increasing percentage of hypsodont (high-crowned teeth) mammals in the Neogene of Europe being due to an enrichment of hyposodont taxa and not a depletion of non-hypsodont taxa.



What these results support is a gradual change to the ecotypic diversity of the regional species pool for the Cenozoic. The rapidity of Cenozoic environmental change is worth discussing. If change is rapid, ecotypic composition of species pool does not seem to track environmental change. If change is gradual then there is the possibility that changes to ecotypic composition may be tracking environmental change.

If plant phase is associated with differences in ecotype occurrence this is interpreted to mean that ecotype enrichment or depletion is due to to associations between that ecotype and whatever plants are dominate at that time.

Temperature affects very few of the occurrence, origination, or survival probabilities of the mammal ecotypes except for a negative relationship between temperature and the origination probabilities of digitigrade carnivores, and both digitigrade and unguligrade herbivores. The origination probabilities and diversity of these three groups all increase over the Cenozoic as average global temperature decreased. This result coupled with the lack of relationship between temperature and the other ecotypes may be responsible for the continued confusion surrounding the impact of temperature on mammal diversity and diversification \citep{Alroy1996a,Alroy2000g,Figueirido2012,Janis1993c,Blois2009}.

What is the comparative size of the effects of plant phase and temperature? Both seem of ``equal'' importance in the sense that they have similar effect sizes on the ecotypes. Perhaps focusing on temperature and not considering other measures of environmental context has been a mistake and perhaps led to increasing confusion in discussions of how ``environment'' effects mammal diversity and diversification. The environment or climate is not just global or regional temperature, it is the set of all possible biotic and abiotic interactions. By including more descriptors of species' environmental context a more complete ``picture'' of the diversification process is inferred.


The effect of species mass on either occurrence or origination and extinction was not allowed to vary by ecotype or environmental context even though it is not known if this is the case or not CITATION. The primary reason for this modeling choice was that this study focuses on ecotypic based differences in either occurrence, or origination and extinction. Allowing the effect of body size to vary by ecotype, time, and environmental factors would increase the overall complexity of the model, something that I felt was not necessary because the overall scope of the study. Instead, body size was included in order to control for its possible underlying effects CITATION. A control means that if there is variation due to body mass, having a term to ``absorb'' that effect is better than ignoring it which may affect other parameter estimates. Additionally, the effect of body size was allowed to have a second-order polynomial form and no higher order polynomials were considered; this was done because it is hard to conceive of a more complex third- or higher-order relationship between body size and the other parameters. Additionally, nonlinearity is rarely if ever considered in the first place, so the simple act of estimating a potential second-order relationship is an opportunity to test more complex hypotheses of the effects of body size on macroevolutionary and macroecological processes.

The only covariate allowed to affect sampling probability was mass and only as a linear predictor. Other covariates, such as the environmental factors considered here, could have affected the underlying preservation process that limits sampling probability. Their exclusion as covariates of sampling/observation was the product of a few key decisions: model complexity, model interpretability, the scope of this study, and a lack of good hypotheses related to these covariates to warrant their inclusion. It should be noted that in other similar studies that use a hidden birth-death model to handle simultaneous estimation of sampling, origination, and extinction have not considered the possible effects of covariates, both species traits and environmental factors, on sampling CITATION.

The time scale available with paleontological data is much greater than that obtainable from modern ecological studies, even long running observations CITATION. Specifically, the temporal scale of paleontological data allows for the complete turnover of a species pool to be observed, something that is impossible in ``real time.'' However, paleontological data is very limited in its spatial resolution, so the analysis of how the ecotypic diversity local communities change over time and how that is also the product of larger scale regional turnover remains unanswered.

The potential effects of common ancestry (i.e. phylogeny) on origination and extinction are not directly considered in this analysis. While a birth-death process approximates the speciation-extinction process underlying the phylogeny \citep{Silvestro2014a} this is not same as considering how the similarity between closely related species may affect the estimates of the effects of species traits to environmental factors on both origination and extinction \citep{Smits2015b,Harnik2014}. One of the principle barriers to the inclusion of the effect of phylogeny in either the pure-presence or birth-death models is computational; with well over 1000 tips, the calculation of the scale parameter defining the phylogenetic effect would be very slow and further increase the already slow computation time necessary for both the marginalization of the discrete occurrence histories and data augmentation already included in both models.

Phylogenetic comparative community ecology and phylogenetic comparative biogeography also discusses how the macroevolutionary processes helps structure an observed community, though it is not necessarily phrased that way. However, that community did not form in isolation but it the result of many factors interacting over time including incumbency, competition, limiting similarity, etc.







\end{document}
