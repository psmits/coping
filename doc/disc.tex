\documentclass[12pt,letterpaper]{article}

\usepackage{amsmath, amsthm}
\usepackage{microtype, parskip}
\usepackage[comma,numbers,sort&compress]{natbib}
\usepackage{lineno}
\usepackage{docmute}
\usepackage{caption, subcaption, multirow, morefloats, rotating}
\usepackage{wrapfig}
\usepackage{attrib}

\frenchspacing

\begin{document}

\section*{Discussion}

Both the composition of a species pool and its environmental context change over time, though not necessarily at the same rate or concurrently. Local communities, whose species are drawn from the regional species pool, have ``roles'' in their communities defined by their interactions with a host of biotic and abiotic interactors (i.e. a species' niche). For higher level ecological characterizations like ecotypes and guilds, these roles are broad and not defined by specific interactions but by the genre of interactions species within that grouping participate in. The diversity of species within an ecotype or guild can be stable over millions of years despite constant species turnover \citep{Jernvall2004,Slater2015c,Valkenburgh1999}. This implies that the size and scope of the role of an ecotype or guild in local communities, and the regional species pool as a whole, is preserved even as the individual interactors change. This also implies that the structure of regional species pools can be constant over time despite a constantly changing set of ``players.'' There is even evidence that functional groups are at least partially self-organizing and truly emergent \citep{Scheffer2006a}.



\subsection*{Conclusions}

These results add a considerable degree of nuance to the narrative of changes to North American diversity being gradual. My results support the conclusions that ecotypic diversity is shaped more by changes to origination than extinction and that major changes to total diversification rate can be attributed to increases in origination of only some ecotypes. There are a number of interesting estimated ecotype diversity patterns. While arboreal ecotypes are diverse in the Paleogene, by the Neogene all arboreal ecotypes dramatically decreased in diversity and became either rare or absent from the regional species pool. The other ecotypes that decrease in diversity over the Cenozoic are plantigrade and scansorial insectivores and scansorial omnivores. The only ecotypes that demonstrate a sustained pattern of increasing diversity are digitigrade and unguligrade herbivores. When the environmental covariates analyzed here are inferred to affect the diversification of an ecotype, this effect is virtually always on origination and not survival. This analysis provides a much more complete picture of North American mammal diversity and diversification, specifically the dynamics of the ecotypic composition of that diversity. By increasing the complexity of analysis while precisely translating research questions into a statistical model, the context of the results is much better understood. Future studies of diversity and diversification should incorporate as much information as possible into their analyses in order to better understand or at least contextualize the complex processes underlying that diversity.





\end{document}
