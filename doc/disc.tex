\documentclass[12pt,letterpaper]{article}

\usepackage{amsmath, amsthm}
\usepackage{microtype, parskip}
\usepackage[comma,numbers,sort&compress]{natbib}
\usepackage{lineno}
\usepackage{docmute}
\usepackage{caption, subcaption, multirow, morefloats, rotating}
\usepackage{wrapfig}

\frenchspacing

\begin{document}

\section*{Discussion}

Regional species pool dynamics cannot simply be described by a single probability of occurrence but instead is the product of both origination and extinction. 

This dimension is not knowable from modern fourth-corner analyses.

Additionally, changes to ecotypic composition of the North American regional species pool are driven primarily by variation in origination rates.



How do the results line up with previous observations and hypotheses discussed in the introduction? 

Extinction rate for the entire regional species pool through time is highly variable and demonstrates a saw-toothed pattern around an apparently constant mean. While a constant mean extinction rate is consistent with previous observation \citep{Alroy1996a,Alroy2000g}, the degree to which extinction rate is actually variable may not have been equally appreciated. What is most consistent with previous observations \citep{Alroy1996a,Alroy2000g}, however, is that diversity seems to be most structured by origination than extinction.

Plant phase always affects plantigrade ecotypes. Does it affect all plantigrade taxa?

Temperature affects very little in general with a few major exceptions: origination probabilities of digitigrade carnivores, and digitigrade and unguligrade herbivores. Why?

Arboreal taxa disappear over the Cenozoic, with massive disappearance by the Paleogene-Neogene barrier. This is consistent with one of the possible explanations presented: Paleogene-Neogene are different and while the earliest Cenozoic may have been neutral wrt arboreal taxa, they disappeared quickly which may account for their higher extinction risk.

Digitigrade carnivores have a relatively stable diversity history through the Cenozoic and could be characterized as varying around a constant mean diversity. This result is consistent with similar observations in \citet{Slater2015c}.

Both digitigrade and unguligrade herbivores increase in diversity over the Cenozoic. The increase of these cursorial forms is consistent.




What these results support is a gradual change to the ecotypic diversity of the regional species pool for the Cenozoic.

The rapidity of Cenozoic environmental change is worth discussing. If change is rapid, ecotypic composition of species pool does not seem to track environmental change. If change is gradual then there is the possibility that changes to ecotypic composition may be tracking environmental change.


Phylogenetic comparative community ecology and phylogenetic comparative biogeography also discusses how the macroevolutionary processes helps structure an observed community, though it is not necessarily phrased that way. However, that community did not form in isolation but it the result of many factors interacting over time including incumbency, competition, limiting similarity, etc.

\end{document}
